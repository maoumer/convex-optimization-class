\documentclass[class=article,crop=false]{standalone} 
%Fall 2020
% Some basic packages
\usepackage{standalone}[subpreambles=true]
\usepackage[utf8]{inputenc}
\usepackage[T1]{fontenc}
\usepackage{textcomp}
\usepackage[english]{babel}
\usepackage{url}
\usepackage{graphicx}
\usepackage{float}
\usepackage{enumitem}
\usepackage{lmodern}
\usepackage{hyperref}
\usepackage[usenames,svgnames,dvipsnames]{xcolor}


\pdfminorversion=7

% Don't indent paragraphs, leave some space between them
\usepackage{parskip}

% Hide page number when page is empty
\usepackage{emptypage}
\usepackage{subcaption}
\usepackage{multicol}
\usepackage[dvipsnames]{xcolor}
\usepackage[b]{esvect}

% Math stuff
\usepackage{amsmath, amsfonts, mathtools, amsthm, amssymb}
\usepackage{bbm}

% Fancy script capitals
\usepackage{mathrsfs}
\usepackage{cancel}
% Bold math
\usepackage{bm}
% Some shortcuts
\newcommand{\rr}{\ensuremath{\mathbb{R}}}
\newcommand{\zz}{\ensuremath{\mathbb{Z}}}
\newcommand{\qq}{\ensuremath{\mathbb{Q}}}
\newcommand{\nn}{\ensuremath{\mathbb{N}}}
\newcommand{\ff}{\ensuremath{\mathbb{F}}}
\newcommand{\cc}{\ensuremath{\mathbb{C}}}
\newcommand{\ee}{\ensuremath{\mathbb{E}}}
\renewcommand\O{\ensuremath{\emptyset}}
\newcommand{\norm}[1]{{\left\lVert{#1}\right\rVert}}
\newcommand{\ve}[1]{{\mathbf{#1}}}
\newcommand\allbold[1]{{\boldmath\textbf{#1}}}
\DeclareMathOperator{\lcm}{lcm}
\DeclareMathOperator{\im}{im}
\DeclareMathOperator{\coim}{coim}
\DeclareMathOperator{\dom}{dom}
\DeclareMathOperator{\tr}{tr}
\DeclareMathOperator{\rank}{rank}
\DeclareMathOperator*{\var}{Var}
\DeclareMathOperator*{\ev}{E}
\DeclareMathOperator{\sinc}{sinc}
\DeclareMathOperator{\dg}{deg}
\DeclareMathOperator{\aff}{aff}
\DeclareMathOperator{\conv}{conv}
\DeclareMathOperator{\epi}{epi}
\DeclareMathOperator{\inte}{int}
\DeclareMathOperator{\ri}{ri}
\DeclareMathOperator*{\argmin}{argmin}
\DeclareMathOperator*{\argmax}{argmax}
\DeclareMathOperator{\graph}{graph}
\DeclareMathOperator{\sgn}{sgn}
\DeclareMathOperator*{\Rep}{Rep}
\DeclareMathOperator{\Proj}{Proj}
\DeclareMathOperator{\prox}{prox}
\DeclareMathOperator{\mat}{mat}
\let\vec\relax
\DeclareMathOperator{\vec}{vec}
\let\Re\relax
\DeclareMathOperator{\Re}{Re}
\let\Im\relax
\DeclareMathOperator{\Im}{Im}
% Put x \to \infty below \lim
\let\svlim\lim\def\lim{\svlim\limits}

%wide hat
\usepackage{scalerel,stackengine}
\stackMath
\newcommand*\wh[1]{%
\savestack{\tmpbox}{\stretchto{%
  \scaleto{%
    \scalerel*[\widthof{\ensuremath{#1}}]{\kern-.6pt\bigwedge\kern-.6pt}%
    {\rule[-\textheight/2]{1ex}{\textheight}}%WIDTH-LIMITED BIG WEDGE
  }{\textheight}% 
}{0.5ex}}%
\stackon[1pt]{#1}{\tmpbox}%
}
\parskip 1ex

%Make implies and impliedby shorter
\let\implies\Rightarrow
\let\impliedby\Leftarrow
\let\iff\Leftrightarrow
\let\epsilon\varepsilon

% Add \contra symbol to denote contradiction
\usepackage{stmaryrd} % for \lightning
\newcommand\contra{\scalebox{1.5}{$\lightning$}}

% \let\phi\varphi

% Command for short corrections
% Usage: 1+1=\correct{3}{2}

\definecolor{correct}{HTML}{009900}
\newcommand\correct[2]{\ensuremath{\:}{\color{red}{#1}}\ensuremath{\to }{\color{correct}{#2}}\ensuremath{\:}}
\newcommand\green[1]{{\color{correct}{#1}}}

% horizontal rule
\newcommand\hr{
    \noindent\rule[0.5ex]{\linewidth}{0.5pt}
}

% hide parts
\newcommand\hide[1]{}

% si unitx
\usepackage{siunitx}
\sisetup{locale = FR}

%allows pmatrix to stretch
\makeatletter
\renewcommand*\env@matrix[1][\arraystretch]{%
  \edef\arraystretch{#1}%
  \hskip -\arraycolsep
  \let\@ifnextchar\new@ifnextchar
  \array{*\c@MaxMatrixCols c}}
\makeatother

\renewcommand{\arraystretch}{0.8}

% Environments
\makeatother
% For box around Definition, Theorem, \ldots
%%fakesection Theorems
\usepackage{thmtools}
\usepackage[framemethod=TikZ]{mdframed}

\theoremstyle{definition}
\mdfdefinestyle{mdbluebox}{%
	roundcorner = 10pt,
	linewidth=1pt,
	skipabove=12pt,
	innerbottommargin=9pt,
	skipbelow=2pt,
	nobreak=true,
	linecolor=blue,
	backgroundcolor=TealBlue!5,
}
\declaretheoremstyle[
	headfont=\sffamily\bfseries\color{MidnightBlue},
	mdframed={style=mdbluebox},
	headpunct={\\[3pt]},
	postheadspace={0pt}
]{thmbluebox}

\mdfdefinestyle{mdredbox}{%
	linewidth=0.5pt,
	skipabove=12pt,
	frametitleaboveskip=5pt,
	frametitlebelowskip=0pt,
	skipbelow=2pt,
	frametitlefont=\bfseries,
	innertopmargin=4pt,
	innerbottommargin=8pt,
	nobreak=false,
	linecolor=RawSienna,
	backgroundcolor=Salmon!5,
}
\declaretheoremstyle[
	headfont=\bfseries\color{RawSienna},
	mdframed={style=mdredbox},
	headpunct={\\[3pt]},
	postheadspace={0pt},
]{thmredbox}

\declaretheorem[%
style=thmbluebox,name=Theorem,numberwithin=section]{thm}
\declaretheorem[style=thmbluebox,name=Lemma,sibling=thm]{lem}
\declaretheorem[style=thmbluebox,name=Proposition,sibling=thm]{prop}
\declaretheorem[style=thmbluebox,name=Corollary,sibling=thm]{coro}
\declaretheorem[style=thmredbox,name=Example,sibling=thm]{eg}

\mdfdefinestyle{mdgreenbox}{%
	roundcorner = 10pt,
	linewidth=1pt,
	skipabove=12pt,
	innerbottommargin=9pt,
	skipbelow=2pt,
	nobreak=true,
	linecolor=ForestGreen,
	backgroundcolor=ForestGreen!5,
}

\declaretheoremstyle[
	headfont=\bfseries\sffamily\color{ForestGreen!70!black},
	bodyfont=\normalfont,
	spaceabove=2pt,
	spacebelow=1pt,
	mdframed={style=mdgreenbox},
	headpunct={ --- },
]{thmgreenbox}

\declaretheorem[style=thmgreenbox,name=Definition,sibling=thm]{defn}

\mdfdefinestyle{mdgreenboxsq}{%
	linewidth=1pt,
	skipabove=12pt,
	innerbottommargin=9pt,
	skipbelow=2pt,
	nobreak=true,
	linecolor=ForestGreen,
	backgroundcolor=ForestGreen!5,
}
\declaretheoremstyle[
	headfont=\bfseries\sffamily\color{ForestGreen!70!black},
	bodyfont=\normalfont,
	spaceabove=2pt,
	spacebelow=1pt,
	mdframed={style=mdgreenboxsq},
	headpunct={},
]{thmgreenboxsq}
\declaretheoremstyle[
	headfont=\bfseries\sffamily\color{ForestGreen!70!black},
	bodyfont=\normalfont,
	spaceabove=2pt,
	spacebelow=1pt,
	mdframed={style=mdgreenboxsq},
	headpunct={},
]{thmgreenboxsq*}

\mdfdefinestyle{mdblackbox}{%
	skipabove=8pt,
	linewidth=3pt,
	rightline=false,
	leftline=true,
	topline=false,
	bottomline=false,
	linecolor=black,
	backgroundcolor=RedViolet!5!gray!5,
}
\declaretheoremstyle[
	headfont=\bfseries,
	bodyfont=\normalfont\small,
	spaceabove=0pt,
	spacebelow=0pt,
	mdframed={style=mdblackbox}
]{thmblackbox}

\theoremstyle{plain}
\declaretheorem[name=Question,sibling=thm,style=thmblackbox]{ques}
\declaretheorem[name=Remark,sibling=thm,style=thmgreenboxsq]{remark}
\declaretheorem[name=Remark,sibling=thm,style=thmgreenboxsq*]{remark*}

\theoremstyle{definition}
\newtheorem{claim}[thm]{Claim}
\theoremstyle{remark}
\newtheorem*{case}{Case}
\newtheorem*{notation}{Notation}
\newtheorem*{note}{Note}
\newtheorem*{motivation}{Motivation}
\newtheorem*{intuition}{Intuition}

% Make section starts with 1 for report type
%\renewcommand\thesection{\arabic{section}}

% End example and intermezzo environments with a small diamond (just like proof
% environments end with a small square)
\usepackage{etoolbox}
\AtEndEnvironment{vb}{\null\hfill$\diamond$}%
\AtEndEnvironment{intermezzo}{\null\hfill$\diamond$}%
% \AtEndEnvironment{opmerking}{\null\hfill$\diamond$}%

% Fix some spacing
% http://tex.stackexchange.com/questions/22119/how-can-i-change-the-spacing-before-theorems-with-amsthm
\makeatletter
\def\thm@space@setup{%
  \thm@preskip=\parskip \thm@postskip=0pt
}

% Fix some stuff
% %http://tex.stackexchange.com/questions/76273/multiple-pdfs-with-page-group-included-in-a-single-page-warning
\pdfsuppresswarningpagegroup=1

\renewcommand{\baselinestretch}{1.5}
\RequirePackage{hyperref}[6.83]
\hypersetup{
  colorlinks=false,
  frenchlinks=false,
  pdfborder={0 0 0},
  naturalnames=false,
  hypertexnames=false,
  breaklinks
}
\urlstyle{same}

\usepackage{graphics}
\usepackage{epstopdf}

%%
%% Add support for color in order to color the hyperlinks.
%% 
\hypersetup{
  colorlinks = true,
  allcolors = siaminlinkcolor,
  urlcolor = siamexlinkcolor,
}
%%fakesection Links
\hypersetup{
    colorlinks,
    linkcolor={red!50!black},
    citecolor={green!50!black},
    urlcolor={blue!80!black}
}
%customization of cleveref
\RequirePackage[capitalize,nameinlink]{cleveref}[0.19]

% Per SIAM Style Manual, "section" should be lowercase
\crefname{section}{section}{sections}
\crefname{subsection}{subsection}{subsections}
\Crefname{section}{Section}{Sections}
\Crefname{subsection}{Subsection}{Subsections}

% Per SIAM Style Manual, "Figure" should be spelled out in references
\Crefname{figure}{Figure}{Figures}

% Per SIAM Style Manual, don't say equation in front on an equation.
\crefformat{equation}{\textup{#2(#1)#3}}
\crefrangeformat{equation}{\textup{#3(#1)#4--#5(#2)#6}}
\crefmultiformat{equation}{\textup{#2(#1)#3}}{ and \textup{#2(#1)#3}}
{, \textup{#2(#1)#3}}{, and \textup{#2(#1)#3}}
\crefrangemultiformat{equation}{\textup{#3(#1)#4--#5(#2)#6}}%
{ and \textup{#3(#1)#4--#5(#2)#6}}{, \textup{#3(#1)#4--#5(#2)#6}}{, and \textup{#3(#1)#4--#5(#2)#6}}

% But spell it out at the beginning of a sentence.
\Crefformat{equation}{#2Equation~\textup{(#1)}#3}
\Crefrangeformat{equation}{Equations~\textup{#3(#1)#4--#5(#2)#6}}
\Crefmultiformat{equation}{Equations~\textup{#2(#1)#3}}{ and \textup{#2(#1)#3}}
{, \textup{#2(#1)#3}}{, and \textup{#2(#1)#3}}
\Crefrangemultiformat{equation}{Equations~\textup{#3(#1)#4--#5(#2)#6}}%
{ and \textup{#3(#1)#4--#5(#2)#6}}{, \textup{#3(#1)#4--#5(#2)#6}}{, and \textup{#3(#1)#4--#5(#2)#6}}

% Make number non-italic in any environment.
\crefdefaultlabelformat{#2\textup{#1}#3}

% My name
\author{Jaden Wang}



\begin{document}
\subsection{Newton's Method revisited, [BV04] Ch.9}
\begin{align*}
	\Delta x_{nt} &= \argmin_{\Delta x} f(x_k) + \langle \nabla f(x_k), \Delta x \rangle + \frac{1}{2} \langle \Delta x|\nabla ^2f(x_k)|\Delta x \rangle\\
		      &= \nabla ^2 f(x_k)^{-1} \nabla f(x_k) && \nabla ^2f(x_k) \succ 0 \\
	x_{k+1} &= x_k+t \Delta x_{nt}, t\approx 1
\end{align*}
Observe: this is affine invariant. That is, let $ \widetilde{ x} = Ax+b$, $ A$ invertible. And the update step doesn't change. This allows us to precondition the Hessian. Thus, convergence ought to be independent of the condition number of Hessian.

\subsubsection{old-fashioned analysis}
Assume $ \mu I \preceq \nabla ^2f(x) \preceq MI \ \forall \ x$, sometimes $ \norm{ \nabla ^3 f(x)} \leq L$. These are strong assumptions, depend on potentially unknown parameters, and not affine-invariant.
\subsubsection{self-concordant analysis}
Introduced by Nesterov and Nemirovski. 

\begin{defn}[self-concordant]
	$ f:\rr \to \rr$ is \allbold{self-concordant} (sc) if $ f$ is convex and
	 \begin{align*}
		 |f'''(x)| \leq 2(f''(x))^{\frac{3}{2}}
	\end{align*}
\end{defn}
\begin{note}
	$ f'''(0) \implies$ sc.
\end{note}
\begin{eg}
	$ f(x) = -\log(x)$. Let's check if $ f$ is sc:  $ f''(x) = \frac{1}{x^2}, f'''(-\frac{2}{x^3})$, and it works!
\end{eg}

\begin{defn}
	$ f: \rr^{n} \to \rr$ is sc if $ \ \forall \ x,v, \phi(t) = f(x+tv)$ is sc.
\end{defn}
\begin{note}
This is just sc for all the lines.
\end{note}
\begin{prop}
	If $ f(x)$ is sc, so is  $ x \mapsto f(Ax+b)$, $ A$ is  $ n \times n$. This is affine-invariant.
\end{prop}
\begin{eg}
	linear or quadratic, $ -\log (\ve{x}) := \sum -\log(x_i)$ is sc. So is $ -\log \det(X)$.
\end{eg}

\begin{prop}
	$ \nabla ^2 f(x) \succ 0 $ and $ f$ is sc  $ \iff$ $ f$ is strictly convex and  $ f$ is sc. 
\end{prop}
\begin{remark}
	Recall that $ \nabla ^2 f(x) \succ 0$ implies strong (thus strict) convexity but strict convexity doesn't imply positive-definite Hessian even if the Hessian exists. It only implies positive-semidefinite Hessian. This proposition allows us to make a stronger claim.
\end{remark}
\subsubsection{damped/guarded Newton method}
suitable for strictly convex functions.

Let $ \Delta x_{nt} = -(\nabla ^2 f_k)^{-1} \nabla f_k$. $ \norm{ z}_{H} := \sqrt{\langle z|H|z \rangle}  , H \succ 0$.
\begin{align*}
	\lambda_k^2 := \lambda^2(x_k) &:= \norm{ \Delta x_{nt}}_{\nabla ^2 f_k}^2 && \text{ Newton decrement} \\
	&:= \langle \nabla f_k| \nabla ^2 f_k^{-1}|\nabla f_k \rangle 
\end{align*}
\begin{remark}
By the proposition above, we ensure $ \nabla ^2 f_k \succ 0$ defining a valid norm.
\end{remark}
terminate if $ \lambda_k^2 /2<$tol

backtracking linesearch to ensure
\begin{align*}
	f(x_k + t \Delta x_{nt}) &< f_k + \underbrace{ \alpha}_{ \in (0,0.5)} \underbrace{ t}_{ t_0=1}  \underbrace{  \langle \nabla f_k, \Delta x_{nt} \rangle}_{ -\lambda_k ^2}\\
	x_{k+1} &= x_k+ t \Delta x_{nt}
\end{align*}

\begin{prop}[BV Ch. 9.6.3]
	Let $ p^*  = \min_{x } f(x)$. If $ \lambda(x) < 0.68$, then $ f(x)-p^* \leq \lambda^2(x)$.
\end{prop}

\begin{thm}
If $ f$ is sc, strictly onvex, etc, there exists  $ 0< \eta < \frac{1}{4}$, $ \gamma > 0$ s.t. 
\begin{enumerate}[label=(\roman*)]
	\item Damped Newton phase: $ \lambda(x_k)> \eta$, then $ f(x_{k+1}) - f(x_k) < -\gamma$
	\item Quadratic convergence phase: $ \lambda(x_k) \leq \eta$, then $ t=1$ and  $ 2 \lambda_{k+1} \leq (2\lambda_k)^2$. Here we have the bound:
		\begin{align*}
			f(x_{k_0+\ell})-p^* \leq \frac{1}{4} \left( \frac{1}{2} \right) ^{2^{\ell-k_0+1}}
		\end{align*}
\end{enumerate}
\end{thm}
\begin{note}
	Before we get to the second phase, it's a constant rate and could be bad if it's too far away, but once we hit the second phase and we get amazing accuracy very quickly (less than 6 iterations).
\end{note}
The total number of iterations to $ \epsilon$-solution is
\begin{align*}
	\frac{f(x_0)-p^* }{ \gamma}+ \log_2 \left( \log_2 \left( \frac{1}{ \epsilon} \right)  \right) \approx 375(f(x_0)-p^* ) + 6
\end{align*}

\subsubsection{Newton's method with equality constraints}
This is easy!

\begin{align*}
\min\quad &f_0(x) \\
&Ax=b
\end{align*}
Then
\begin{align*}
	\mathscr{L}(x,\nu) = f(x) + \nu^{T}(Ax-b)
\end{align*}
The KKT conditions are:
\begin{enumerate}[label=(\arabic*)]
	\item $ 0=\nabla f(x) + A^{T} \nu$.
	\item $ Ax=b$.
\end{enumerate}
These are already linearized. So the update step with Taylor expansion becomes:
\begin{align*}
	0&= \nabla f_k + \nabla ^2 f_k \Delta x + A^{T} \nu \\
	b&= A(x+ \Delta x)
\end{align*}
\begin{align*}
	\underbrace{ \begin{pmatrix} \nabla ^2 f_k& A^{T}\\A&0 \end{pmatrix}  }_{ \text{ saddle pt system}}\begin{pmatrix} \Delta x\\ \nu \end{pmatrix} = \begin{pmatrix} -\nabla f_k \\ -A x_k \end{pmatrix}   
\end{align*}

Another way to think is that $ x = Fz + x_p$, where  $ Fz \in \ker A$. Then
\begin{align*}
	\min_{z} f(Fz+x_p)
\end{align*}
Affine-invariant makes this easy using Newton. Then this problem is equivalent to the saddle point system via Schur's complement.
\subsubsection{Newton with inequality constraints: IPM}
\begin{align*}
\min\quad &f_0(x) \\
\text{subject to } \quad &f_i(x) \leq 0, i = 1,\ldots,m \\
			 & Ax=b
\end{align*}
Using log barrier, we change the inequality constraints into solving the sc function
\begin{align*}
	x^* (t) &= \argmin_{x, Ax=b} f_0(x) +  \underbrace{ \sum_{ i= 1}^{ m} -\frac{1}{t} \log(-f_i(x)) }_{ \phi_t(x), \text{ sc} }
\end{align*}
Central path: $ \{x^* (t)\}_{t \geq 0} $. As we increase $ t$, we get closer to the boundary (think about the log barrier plot). If we choose $ t$ to be too large, it will take a long time. Instead we solve for an increasing sequence of  $ t$ using warm-start using solution from previous $ t$. 

The primal-dual method take one step of Newton at each $ t$, and might be more efficient.

$ f(x_k) - p^* \leq f(x_k) - g( \lambda_k, \nu_k) = \frac{m}{t}$. We can prove convergence using this nicely. See BV04 Ch 11 for more details on IPM.
\end{document}
