\documentclass[class=article,crop=false]{standalone} 
%Fall 2020
% Some basic packages
\usepackage{standalone}[subpreambles=true]
\usepackage[utf8]{inputenc}
\usepackage[T1]{fontenc}
\usepackage{textcomp}
\usepackage[english]{babel}
\usepackage{url}
\usepackage{graphicx}
\usepackage{float}
\usepackage{enumitem}
\usepackage{lmodern}
\usepackage{hyperref}
\usepackage[usenames,svgnames,dvipsnames]{xcolor}


\pdfminorversion=7

% Don't indent paragraphs, leave some space between them
\usepackage{parskip}

% Hide page number when page is empty
\usepackage{emptypage}
\usepackage{subcaption}
\usepackage{multicol}
\usepackage[dvipsnames]{xcolor}
\usepackage[b]{esvect}

% Math stuff
\usepackage{amsmath, amsfonts, mathtools, amsthm, amssymb}
\usepackage{bbm}

% Fancy script capitals
\usepackage{mathrsfs}
\usepackage{cancel}
% Bold math
\usepackage{bm}
% Some shortcuts
\newcommand{\rr}{\ensuremath{\mathbb{R}}}
\newcommand{\zz}{\ensuremath{\mathbb{Z}}}
\newcommand{\qq}{\ensuremath{\mathbb{Q}}}
\newcommand{\nn}{\ensuremath{\mathbb{N}}}
\newcommand{\ff}{\ensuremath{\mathbb{F}}}
\newcommand{\cc}{\ensuremath{\mathbb{C}}}
\newcommand{\ee}{\ensuremath{\mathbb{E}}}
\renewcommand\O{\ensuremath{\emptyset}}
\newcommand{\norm}[1]{{\left\lVert{#1}\right\rVert}}
\newcommand{\ve}[1]{{\mathbf{#1}}}
\newcommand\allbold[1]{{\boldmath\textbf{#1}}}
\DeclareMathOperator{\lcm}{lcm}
\DeclareMathOperator{\im}{im}
\DeclareMathOperator{\coim}{coim}
\DeclareMathOperator{\dom}{dom}
\DeclareMathOperator{\tr}{tr}
\DeclareMathOperator{\rank}{rank}
\DeclareMathOperator*{\var}{Var}
\DeclareMathOperator*{\ev}{E}
\DeclareMathOperator{\sinc}{sinc}
\DeclareMathOperator{\dg}{deg}
\DeclareMathOperator{\aff}{aff}
\DeclareMathOperator{\conv}{conv}
\DeclareMathOperator{\epi}{epi}
\DeclareMathOperator{\inte}{int}
\DeclareMathOperator{\ri}{ri}
\DeclareMathOperator*{\argmin}{argmin}
\DeclareMathOperator*{\argmax}{argmax}
\DeclareMathOperator{\graph}{graph}
\DeclareMathOperator{\sgn}{sgn}
\DeclareMathOperator*{\Rep}{Rep}
\DeclareMathOperator{\Proj}{Proj}
\DeclareMathOperator{\prox}{prox}
\DeclareMathOperator{\mat}{mat}
\let\vec\relax
\DeclareMathOperator{\vec}{vec}
\let\Re\relax
\DeclareMathOperator{\Re}{Re}
\let\Im\relax
\DeclareMathOperator{\Im}{Im}
% Put x \to \infty below \lim
\let\svlim\lim\def\lim{\svlim\limits}

%wide hat
\usepackage{scalerel,stackengine}
\stackMath
\newcommand*\wh[1]{%
\savestack{\tmpbox}{\stretchto{%
  \scaleto{%
    \scalerel*[\widthof{\ensuremath{#1}}]{\kern-.6pt\bigwedge\kern-.6pt}%
    {\rule[-\textheight/2]{1ex}{\textheight}}%WIDTH-LIMITED BIG WEDGE
  }{\textheight}% 
}{0.5ex}}%
\stackon[1pt]{#1}{\tmpbox}%
}
\parskip 1ex

%Make implies and impliedby shorter
\let\implies\Rightarrow
\let\impliedby\Leftarrow
\let\iff\Leftrightarrow
\let\epsilon\varepsilon

% Add \contra symbol to denote contradiction
\usepackage{stmaryrd} % for \lightning
\newcommand\contra{\scalebox{1.5}{$\lightning$}}

% \let\phi\varphi

% Command for short corrections
% Usage: 1+1=\correct{3}{2}

\definecolor{correct}{HTML}{009900}
\newcommand\correct[2]{\ensuremath{\:}{\color{red}{#1}}\ensuremath{\to }{\color{correct}{#2}}\ensuremath{\:}}
\newcommand\green[1]{{\color{correct}{#1}}}

% horizontal rule
\newcommand\hr{
    \noindent\rule[0.5ex]{\linewidth}{0.5pt}
}

% hide parts
\newcommand\hide[1]{}

% si unitx
\usepackage{siunitx}
\sisetup{locale = FR}

%allows pmatrix to stretch
\makeatletter
\renewcommand*\env@matrix[1][\arraystretch]{%
  \edef\arraystretch{#1}%
  \hskip -\arraycolsep
  \let\@ifnextchar\new@ifnextchar
  \array{*\c@MaxMatrixCols c}}
\makeatother

\renewcommand{\arraystretch}{0.8}

% Environments
\makeatother
% For box around Definition, Theorem, \ldots
%%fakesection Theorems
\usepackage{thmtools}
\usepackage[framemethod=TikZ]{mdframed}

\theoremstyle{definition}
\mdfdefinestyle{mdbluebox}{%
	roundcorner = 10pt,
	linewidth=1pt,
	skipabove=12pt,
	innerbottommargin=9pt,
	skipbelow=2pt,
	nobreak=true,
	linecolor=blue,
	backgroundcolor=TealBlue!5,
}
\declaretheoremstyle[
	headfont=\sffamily\bfseries\color{MidnightBlue},
	mdframed={style=mdbluebox},
	headpunct={\\[3pt]},
	postheadspace={0pt}
]{thmbluebox}

\mdfdefinestyle{mdredbox}{%
	linewidth=0.5pt,
	skipabove=12pt,
	frametitleaboveskip=5pt,
	frametitlebelowskip=0pt,
	skipbelow=2pt,
	frametitlefont=\bfseries,
	innertopmargin=4pt,
	innerbottommargin=8pt,
	nobreak=false,
	linecolor=RawSienna,
	backgroundcolor=Salmon!5,
}
\declaretheoremstyle[
	headfont=\bfseries\color{RawSienna},
	mdframed={style=mdredbox},
	headpunct={\\[3pt]},
	postheadspace={0pt},
]{thmredbox}

\declaretheorem[%
style=thmbluebox,name=Theorem,numberwithin=section]{thm}
\declaretheorem[style=thmbluebox,name=Lemma,sibling=thm]{lem}
\declaretheorem[style=thmbluebox,name=Proposition,sibling=thm]{prop}
\declaretheorem[style=thmbluebox,name=Corollary,sibling=thm]{coro}
\declaretheorem[style=thmredbox,name=Example,sibling=thm]{eg}

\mdfdefinestyle{mdgreenbox}{%
	roundcorner = 10pt,
	linewidth=1pt,
	skipabove=12pt,
	innerbottommargin=9pt,
	skipbelow=2pt,
	nobreak=true,
	linecolor=ForestGreen,
	backgroundcolor=ForestGreen!5,
}

\declaretheoremstyle[
	headfont=\bfseries\sffamily\color{ForestGreen!70!black},
	bodyfont=\normalfont,
	spaceabove=2pt,
	spacebelow=1pt,
	mdframed={style=mdgreenbox},
	headpunct={ --- },
]{thmgreenbox}

\declaretheorem[style=thmgreenbox,name=Definition,sibling=thm]{defn}

\mdfdefinestyle{mdgreenboxsq}{%
	linewidth=1pt,
	skipabove=12pt,
	innerbottommargin=9pt,
	skipbelow=2pt,
	nobreak=true,
	linecolor=ForestGreen,
	backgroundcolor=ForestGreen!5,
}
\declaretheoremstyle[
	headfont=\bfseries\sffamily\color{ForestGreen!70!black},
	bodyfont=\normalfont,
	spaceabove=2pt,
	spacebelow=1pt,
	mdframed={style=mdgreenboxsq},
	headpunct={},
]{thmgreenboxsq}
\declaretheoremstyle[
	headfont=\bfseries\sffamily\color{ForestGreen!70!black},
	bodyfont=\normalfont,
	spaceabove=2pt,
	spacebelow=1pt,
	mdframed={style=mdgreenboxsq},
	headpunct={},
]{thmgreenboxsq*}

\mdfdefinestyle{mdblackbox}{%
	skipabove=8pt,
	linewidth=3pt,
	rightline=false,
	leftline=true,
	topline=false,
	bottomline=false,
	linecolor=black,
	backgroundcolor=RedViolet!5!gray!5,
}
\declaretheoremstyle[
	headfont=\bfseries,
	bodyfont=\normalfont\small,
	spaceabove=0pt,
	spacebelow=0pt,
	mdframed={style=mdblackbox}
]{thmblackbox}

\theoremstyle{plain}
\declaretheorem[name=Question,sibling=thm,style=thmblackbox]{ques}
\declaretheorem[name=Remark,sibling=thm,style=thmgreenboxsq]{remark}
\declaretheorem[name=Remark,sibling=thm,style=thmgreenboxsq*]{remark*}

\theoremstyle{definition}
\newtheorem{claim}[thm]{Claim}
\theoremstyle{remark}
\newtheorem*{case}{Case}
\newtheorem*{notation}{Notation}
\newtheorem*{note}{Note}
\newtheorem*{motivation}{Motivation}
\newtheorem*{intuition}{Intuition}

% Make section starts with 1 for report type
%\renewcommand\thesection{\arabic{section}}

% End example and intermezzo environments with a small diamond (just like proof
% environments end with a small square)
\usepackage{etoolbox}
\AtEndEnvironment{vb}{\null\hfill$\diamond$}%
\AtEndEnvironment{intermezzo}{\null\hfill$\diamond$}%
% \AtEndEnvironment{opmerking}{\null\hfill$\diamond$}%

% Fix some spacing
% http://tex.stackexchange.com/questions/22119/how-can-i-change-the-spacing-before-theorems-with-amsthm
\makeatletter
\def\thm@space@setup{%
  \thm@preskip=\parskip \thm@postskip=0pt
}

% Fix some stuff
% %http://tex.stackexchange.com/questions/76273/multiple-pdfs-with-page-group-included-in-a-single-page-warning
\pdfsuppresswarningpagegroup=1

\renewcommand{\baselinestretch}{1.5}
\RequirePackage{hyperref}[6.83]
\hypersetup{
  colorlinks=false,
  frenchlinks=false,
  pdfborder={0 0 0},
  naturalnames=false,
  hypertexnames=false,
  breaklinks
}
\urlstyle{same}

\usepackage{graphics}
\usepackage{epstopdf}

%%
%% Add support for color in order to color the hyperlinks.
%% 
\hypersetup{
  colorlinks = true,
  allcolors = siaminlinkcolor,
  urlcolor = siamexlinkcolor,
}
%%fakesection Links
\hypersetup{
    colorlinks,
    linkcolor={red!50!black},
    citecolor={green!50!black},
    urlcolor={blue!80!black}
}
%customization of cleveref
\RequirePackage[capitalize,nameinlink]{cleveref}[0.19]

% Per SIAM Style Manual, "section" should be lowercase
\crefname{section}{section}{sections}
\crefname{subsection}{subsection}{subsections}
\Crefname{section}{Section}{Sections}
\Crefname{subsection}{Subsection}{Subsections}

% Per SIAM Style Manual, "Figure" should be spelled out in references
\Crefname{figure}{Figure}{Figures}

% Per SIAM Style Manual, don't say equation in front on an equation.
\crefformat{equation}{\textup{#2(#1)#3}}
\crefrangeformat{equation}{\textup{#3(#1)#4--#5(#2)#6}}
\crefmultiformat{equation}{\textup{#2(#1)#3}}{ and \textup{#2(#1)#3}}
{, \textup{#2(#1)#3}}{, and \textup{#2(#1)#3}}
\crefrangemultiformat{equation}{\textup{#3(#1)#4--#5(#2)#6}}%
{ and \textup{#3(#1)#4--#5(#2)#6}}{, \textup{#3(#1)#4--#5(#2)#6}}{, and \textup{#3(#1)#4--#5(#2)#6}}

% But spell it out at the beginning of a sentence.
\Crefformat{equation}{#2Equation~\textup{(#1)}#3}
\Crefrangeformat{equation}{Equations~\textup{#3(#1)#4--#5(#2)#6}}
\Crefmultiformat{equation}{Equations~\textup{#2(#1)#3}}{ and \textup{#2(#1)#3}}
{, \textup{#2(#1)#3}}{, and \textup{#2(#1)#3}}
\Crefrangemultiformat{equation}{Equations~\textup{#3(#1)#4--#5(#2)#6}}%
{ and \textup{#3(#1)#4--#5(#2)#6}}{, \textup{#3(#1)#4--#5(#2)#6}}{, and \textup{#3(#1)#4--#5(#2)#6}}

% Make number non-italic in any environment.
\crefdefaultlabelformat{#2\textup{#1}#3}

% My name
\author{Jaden Wang}



\begin{document}
\newpage
\section{Convex Functions [BV04 Ch.3]}

~\begin{defn}[convex function]
	A function $ f: \rr^{n} \to \rr$ is \allbold{convex} if $ \dom(f)$ is a convex set and for all $ x,y \in \dom(f)$ and $ 0\leq t \leq 1$,
	 \[
		 f(tx + (1-t)y)\leq tf(x) + (1-t)f(y)
	.\] 
	It is \allbold{strictly convex} if it has strict inequality. It is \allbold{strongly convex w.r.t. the norm $ \norm{ \cdot } $ with parameter $ \mu$ } if $ \dom(f)$ is a convex set and, for all $ x,y \in \dom(f), x\neq y, 0\leq t \leq 1$,
	\[
		f(tx + (1-t)y) \leq tf(x) + (1-t)f(y) - \frac{\mu}{2} t(1-t) \norm{ x-y}^2 
	.\] 
\end{defn}
\begin{intuition}
	For strongly convex, we can see that when $ t=\frac{1}{2}$ or when the point is midway between $ x,y$, the bound of inequality is a lot smaller than when  $ t$ is closer to either point. This forces the function to have a large curvature.
\end{intuition}

\begin{thm}[simpler characterizations]
	$ f$ is convex if  $ \epi (f)$ is convex ($\implies \dom(f)$ is convex).
	
	$ f$ is strictly convex means it always has curvature and no straight lines.

	$ f$ is strongly convex with parameter  $ \mu$ and w.r.t. $ \norm{ \cdot }_2 $ iff $ x \mapsto f(x) - \frac{\mu}{2} \norm{ x}_2^2 $ is convex (not true for general norms). 
\end{thm}
\begin{note}
Subtraction of convex function doesn't preserve convexity, except in the case of strongly convex w.r.t. Euclidean norm.
\end{note}

\begin{remark}
Convexity is a \emph{global property}. This contrasts with continuity which is a local property.  
\end{remark}

\begin{remark}
	In convex analysis, we allow the \emph{extended value function} $ f: \mathcal{H} \to [-\infty,\infty]$ or $ f: \mathcal{H} \to (-\infty,\infty]$, where $ \mathcal{ H}$ is a generic Hilbert space. 

	This way, if $ x \in \dom(f)$, we can pretend it is but define $ f(x) = +\infty$. This wouldn't affect our minimization problem. Now we can redefine
	\[
		\dom(f) = \{x: f(x)< + \infty\} 
	.\] 
	This will turn out to be convenient for minimization.
\end{remark}
\begin{eg}
Define the \emph{indicator function} of a set $ C$ to be
 \begin{equation*}
	 I_C(x) =
\begin{cases}
	0,& x \in C\\
	+\infty,& x \not\in C
\end{cases}
\end{equation*}

This is different than how we usually define indicator function. Now we can do the following:
\[
	\min_{x \in C} f(x) \iff \min_{x \in \rr^{n}} f(x) + I_C(x)
.\] 
That is, we can turn a constrained minimization problem into an unconstrained problem with huge penalty on going outside the constraint. 
\end{eg}

\begin{defn}[proper function]
	$ f: \rr^{n} \to [-\infty,\infty]$ is \allbold{proper} if
	\begin{enumerate}[label=\arabic*)]
		\item it never takes the value $ -\infty$.
		\item $ \dom(f) \neq \O$. That is, the value doesn't always equal to $ + \infty$.
	\end{enumerate}
\end{defn}
\begin{note}
This way we can assume there exist feasible points, hence "proper".
\end{note}
\begin{eg}
$ I_C$ is proper iff  $ C \neq \O$.
\end{eg}

\begin{defn}[lower semi-continuous (lsc)] 
	$ f: \rr^{n} \to [-\infty,\infty]$ is \allbold{lower semi-continuous (lsc)} at $ x \in \rr^{n}$ if for all $ (x_n)$ s.t. $ x_n \to x$,
	\[
		f(x) \leq \liminf_n f(x_n) \coloneqq \lim_{ n \to \infty} \inf_{n\leq k} f(x_k)
	.\] 
\end{defn}
\begin{intuition}
This is a lot like sequential definition of continuity, except that we allow liminf to go higher than the function.
\end{intuition}

\begin{defn}[lsc function]
$ f$ is a \allbold{lsc function} if $ f$ is lsc for all points  $ x \in \rr^{n}$. 
\end{defn}
~\begin{figure}[H]
	\centering
	\includegraphics[width=\textwidth]{./figures/lsc.png}
\end{figure}
\begin{thm}
	In $ \rr^{n}$ or any Hausdorff space, $ f$ is lsc iff $ \epi(f)$ is a closed set. 
\end{thm}

~\begin{figure}[H]
	\centering
	\includegraphics[width=\textwidth]{./figures/lsc_epi.png}
\end{figure}

\begin{eg}
$ I_C$ is lsc iff  $ C$ is a closed set.
\end{eg}

We can extend classical theorems involving continuity such as the Extreme Value Theorem to lsc.
\begin{thm}
If $ C$ is compact, then  $ f$ is lsc  $ \implies$ $ f$ achieves its minimum over  $ C$.
\end{thm}

\begin{remark}
$ f$ is continuous iff  $ f$ is lsc and usc.
\end{remark}
\begin{notation}
	$ \Gamma(\rr^{n})$ is the set of all lsc and convex functions $ f: \rr^{n} \to [-\infty,\infty]$.

	$ \Gamma_0 (\rr^{n}) \subseteq \Gamma (\rr^{n})$ consists of such functions that is also proper, $ f: \rr^{n} \to (-\infty,\infty]$.

	So $ \Gamma_0(\rr^{n})$ is the standard class of functions for convex optimization.
\end{notation}

\begin{eg}
	$ I_C \in \Gamma_0 (\rr^{n})$ for some $ C \subseteq \rr^{n}$ iff $ C$ is nonempty, closed, and convex.
\end{eg}

\begin{remark}
Recall that the restriction to proper function is mild.

What about restricting to lsc functions? It's also mild in the context of convex functions, because "weird things with convex functions can only involve boundaries (and $ +\infty$).
\end{remark}
\begin{thm}[8.38 BC17 ]
	If $ f: \mathcal{ H} \to (-\infty,\infty]$ is proper and convex, then $ f$ is continuous at  $ x \in \dom(f)$ iff $ f$ is bounded above on a neighborhood of  $ x$.
\end{thm}
\begin{note}
For convex functions, we won't see jumps like in the lsc case.
\end{note}
\begin{coro}[8.39]
Given the same setup, if one of the following is true:
\begin{enumerate}[label=(\roman*)]
	\item $ f$ is bounded above on some neighborhood of $ x$.
	\item  $ f$ is lsc.
	\item  $ \mathcal{ H}$ is finite dimensional.
\end{enumerate}
then $ f$ is continuous on the interior of its domain, $ \inte (\dom(f))$.
\end{coro}
\begin{note}
Under these assumptions, weird discontinuous things can only happen at the boundary.
\end{note}
~\begin{figure}[H]
	\centering
	\includegraphics[width=0.4\textwidth]{./figures/proper_cvx.png}
	\caption{An example of a proper, convex function (not lsc) that isn't continuous due to discontinuity at the boundary. It is however continuous on the interior.}
\end{figure}
\begin{remark}
	In summary, by the corollary if $ f: \rr^{n} \to \rr$ (has full-domain and not equal to $ \pm \infty$), then convex $ \implies$ continuous.
\end{remark}
\end{document}
