\documentclass[class=article,crop=false]{standalone} 
%Fall 2020
% Some basic packages
\usepackage{standalone}[subpreambles=true]
\usepackage[utf8]{inputenc}
\usepackage[T1]{fontenc}
\usepackage{textcomp}
\usepackage[english]{babel}
\usepackage{url}
\usepackage{graphicx}
\usepackage{float}
\usepackage{enumitem}
\usepackage{lmodern}
\usepackage{hyperref}
\usepackage[usenames,svgnames,dvipsnames]{xcolor}


\pdfminorversion=7

% Don't indent paragraphs, leave some space between them
\usepackage{parskip}

% Hide page number when page is empty
\usepackage{emptypage}
\usepackage{subcaption}
\usepackage{multicol}
\usepackage[dvipsnames]{xcolor}
\usepackage[b]{esvect}

% Math stuff
\usepackage{amsmath, amsfonts, mathtools, amsthm, amssymb}
\usepackage{bbm}

% Fancy script capitals
\usepackage{mathrsfs}
\usepackage{cancel}
% Bold math
\usepackage{bm}
% Some shortcuts
\newcommand{\rr}{\ensuremath{\mathbb{R}}}
\newcommand{\zz}{\ensuremath{\mathbb{Z}}}
\newcommand{\qq}{\ensuremath{\mathbb{Q}}}
\newcommand{\nn}{\ensuremath{\mathbb{N}}}
\newcommand{\ff}{\ensuremath{\mathbb{F}}}
\newcommand{\cc}{\ensuremath{\mathbb{C}}}
\newcommand{\ee}{\ensuremath{\mathbb{E}}}
\renewcommand\O{\ensuremath{\emptyset}}
\newcommand{\norm}[1]{{\left\lVert{#1}\right\rVert}}
\newcommand{\ve}[1]{{\mathbf{#1}}}
\newcommand\allbold[1]{{\boldmath\textbf{#1}}}
\DeclareMathOperator{\lcm}{lcm}
\DeclareMathOperator{\im}{im}
\DeclareMathOperator{\coim}{coim}
\DeclareMathOperator{\dom}{dom}
\DeclareMathOperator{\tr}{tr}
\DeclareMathOperator{\rank}{rank}
\DeclareMathOperator*{\var}{Var}
\DeclareMathOperator*{\ev}{E}
\DeclareMathOperator{\sinc}{sinc}
\DeclareMathOperator{\dg}{deg}
\DeclareMathOperator{\aff}{aff}
\DeclareMathOperator{\conv}{conv}
\DeclareMathOperator{\epi}{epi}
\DeclareMathOperator{\inte}{int}
\DeclareMathOperator{\ri}{ri}
\DeclareMathOperator*{\argmin}{argmin}
\DeclareMathOperator*{\argmax}{argmax}
\DeclareMathOperator{\graph}{graph}
\DeclareMathOperator{\sgn}{sgn}
\DeclareMathOperator*{\Rep}{Rep}
\DeclareMathOperator{\Proj}{Proj}
\DeclareMathOperator{\prox}{prox}
\DeclareMathOperator{\mat}{mat}
\let\vec\relax
\DeclareMathOperator{\vec}{vec}
\let\Re\relax
\DeclareMathOperator{\Re}{Re}
\let\Im\relax
\DeclareMathOperator{\Im}{Im}
% Put x \to \infty below \lim
\let\svlim\lim\def\lim{\svlim\limits}

%wide hat
\usepackage{scalerel,stackengine}
\stackMath
\newcommand*\wh[1]{%
\savestack{\tmpbox}{\stretchto{%
  \scaleto{%
    \scalerel*[\widthof{\ensuremath{#1}}]{\kern-.6pt\bigwedge\kern-.6pt}%
    {\rule[-\textheight/2]{1ex}{\textheight}}%WIDTH-LIMITED BIG WEDGE
  }{\textheight}% 
}{0.5ex}}%
\stackon[1pt]{#1}{\tmpbox}%
}
\parskip 1ex

%Make implies and impliedby shorter
\let\implies\Rightarrow
\let\impliedby\Leftarrow
\let\iff\Leftrightarrow
\let\epsilon\varepsilon

% Add \contra symbol to denote contradiction
\usepackage{stmaryrd} % for \lightning
\newcommand\contra{\scalebox{1.5}{$\lightning$}}

% \let\phi\varphi

% Command for short corrections
% Usage: 1+1=\correct{3}{2}

\definecolor{correct}{HTML}{009900}
\newcommand\correct[2]{\ensuremath{\:}{\color{red}{#1}}\ensuremath{\to }{\color{correct}{#2}}\ensuremath{\:}}
\newcommand\green[1]{{\color{correct}{#1}}}

% horizontal rule
\newcommand\hr{
    \noindent\rule[0.5ex]{\linewidth}{0.5pt}
}

% hide parts
\newcommand\hide[1]{}

% si unitx
\usepackage{siunitx}
\sisetup{locale = FR}

%allows pmatrix to stretch
\makeatletter
\renewcommand*\env@matrix[1][\arraystretch]{%
  \edef\arraystretch{#1}%
  \hskip -\arraycolsep
  \let\@ifnextchar\new@ifnextchar
  \array{*\c@MaxMatrixCols c}}
\makeatother

\renewcommand{\arraystretch}{0.8}

% Environments
\makeatother
% For box around Definition, Theorem, \ldots
%%fakesection Theorems
\usepackage{thmtools}
\usepackage[framemethod=TikZ]{mdframed}

\theoremstyle{definition}
\mdfdefinestyle{mdbluebox}{%
	roundcorner = 10pt,
	linewidth=1pt,
	skipabove=12pt,
	innerbottommargin=9pt,
	skipbelow=2pt,
	nobreak=true,
	linecolor=blue,
	backgroundcolor=TealBlue!5,
}
\declaretheoremstyle[
	headfont=\sffamily\bfseries\color{MidnightBlue},
	mdframed={style=mdbluebox},
	headpunct={\\[3pt]},
	postheadspace={0pt}
]{thmbluebox}

\mdfdefinestyle{mdredbox}{%
	linewidth=0.5pt,
	skipabove=12pt,
	frametitleaboveskip=5pt,
	frametitlebelowskip=0pt,
	skipbelow=2pt,
	frametitlefont=\bfseries,
	innertopmargin=4pt,
	innerbottommargin=8pt,
	nobreak=false,
	linecolor=RawSienna,
	backgroundcolor=Salmon!5,
}
\declaretheoremstyle[
	headfont=\bfseries\color{RawSienna},
	mdframed={style=mdredbox},
	headpunct={\\[3pt]},
	postheadspace={0pt},
]{thmredbox}

\declaretheorem[%
style=thmbluebox,name=Theorem,numberwithin=section]{thm}
\declaretheorem[style=thmbluebox,name=Lemma,sibling=thm]{lem}
\declaretheorem[style=thmbluebox,name=Proposition,sibling=thm]{prop}
\declaretheorem[style=thmbluebox,name=Corollary,sibling=thm]{coro}
\declaretheorem[style=thmredbox,name=Example,sibling=thm]{eg}

\mdfdefinestyle{mdgreenbox}{%
	roundcorner = 10pt,
	linewidth=1pt,
	skipabove=12pt,
	innerbottommargin=9pt,
	skipbelow=2pt,
	nobreak=true,
	linecolor=ForestGreen,
	backgroundcolor=ForestGreen!5,
}

\declaretheoremstyle[
	headfont=\bfseries\sffamily\color{ForestGreen!70!black},
	bodyfont=\normalfont,
	spaceabove=2pt,
	spacebelow=1pt,
	mdframed={style=mdgreenbox},
	headpunct={ --- },
]{thmgreenbox}

\declaretheorem[style=thmgreenbox,name=Definition,sibling=thm]{defn}

\mdfdefinestyle{mdgreenboxsq}{%
	linewidth=1pt,
	skipabove=12pt,
	innerbottommargin=9pt,
	skipbelow=2pt,
	nobreak=true,
	linecolor=ForestGreen,
	backgroundcolor=ForestGreen!5,
}
\declaretheoremstyle[
	headfont=\bfseries\sffamily\color{ForestGreen!70!black},
	bodyfont=\normalfont,
	spaceabove=2pt,
	spacebelow=1pt,
	mdframed={style=mdgreenboxsq},
	headpunct={},
]{thmgreenboxsq}
\declaretheoremstyle[
	headfont=\bfseries\sffamily\color{ForestGreen!70!black},
	bodyfont=\normalfont,
	spaceabove=2pt,
	spacebelow=1pt,
	mdframed={style=mdgreenboxsq},
	headpunct={},
]{thmgreenboxsq*}

\mdfdefinestyle{mdblackbox}{%
	skipabove=8pt,
	linewidth=3pt,
	rightline=false,
	leftline=true,
	topline=false,
	bottomline=false,
	linecolor=black,
	backgroundcolor=RedViolet!5!gray!5,
}
\declaretheoremstyle[
	headfont=\bfseries,
	bodyfont=\normalfont\small,
	spaceabove=0pt,
	spacebelow=0pt,
	mdframed={style=mdblackbox}
]{thmblackbox}

\theoremstyle{plain}
\declaretheorem[name=Question,sibling=thm,style=thmblackbox]{ques}
\declaretheorem[name=Remark,sibling=thm,style=thmgreenboxsq]{remark}
\declaretheorem[name=Remark,sibling=thm,style=thmgreenboxsq*]{remark*}

\theoremstyle{definition}
\newtheorem{claim}[thm]{Claim}
\theoremstyle{remark}
\newtheorem*{case}{Case}
\newtheorem*{notation}{Notation}
\newtheorem*{note}{Note}
\newtheorem*{motivation}{Motivation}
\newtheorem*{intuition}{Intuition}

% Make section starts with 1 for report type
%\renewcommand\thesection{\arabic{section}}

% End example and intermezzo environments with a small diamond (just like proof
% environments end with a small square)
\usepackage{etoolbox}
\AtEndEnvironment{vb}{\null\hfill$\diamond$}%
\AtEndEnvironment{intermezzo}{\null\hfill$\diamond$}%
% \AtEndEnvironment{opmerking}{\null\hfill$\diamond$}%

% Fix some spacing
% http://tex.stackexchange.com/questions/22119/how-can-i-change-the-spacing-before-theorems-with-amsthm
\makeatletter
\def\thm@space@setup{%
  \thm@preskip=\parskip \thm@postskip=0pt
}

% Fix some stuff
% %http://tex.stackexchange.com/questions/76273/multiple-pdfs-with-page-group-included-in-a-single-page-warning
\pdfsuppresswarningpagegroup=1

\renewcommand{\baselinestretch}{1.5}
\RequirePackage{hyperref}[6.83]
\hypersetup{
  colorlinks=false,
  frenchlinks=false,
  pdfborder={0 0 0},
  naturalnames=false,
  hypertexnames=false,
  breaklinks
}
\urlstyle{same}

\usepackage{graphics}
\usepackage{epstopdf}

%%
%% Add support for color in order to color the hyperlinks.
%% 
\hypersetup{
  colorlinks = true,
  allcolors = siaminlinkcolor,
  urlcolor = siamexlinkcolor,
}
%%fakesection Links
\hypersetup{
    colorlinks,
    linkcolor={red!50!black},
    citecolor={green!50!black},
    urlcolor={blue!80!black}
}
%customization of cleveref
\RequirePackage[capitalize,nameinlink]{cleveref}[0.19]

% Per SIAM Style Manual, "section" should be lowercase
\crefname{section}{section}{sections}
\crefname{subsection}{subsection}{subsections}
\Crefname{section}{Section}{Sections}
\Crefname{subsection}{Subsection}{Subsections}

% Per SIAM Style Manual, "Figure" should be spelled out in references
\Crefname{figure}{Figure}{Figures}

% Per SIAM Style Manual, don't say equation in front on an equation.
\crefformat{equation}{\textup{#2(#1)#3}}
\crefrangeformat{equation}{\textup{#3(#1)#4--#5(#2)#6}}
\crefmultiformat{equation}{\textup{#2(#1)#3}}{ and \textup{#2(#1)#3}}
{, \textup{#2(#1)#3}}{, and \textup{#2(#1)#3}}
\crefrangemultiformat{equation}{\textup{#3(#1)#4--#5(#2)#6}}%
{ and \textup{#3(#1)#4--#5(#2)#6}}{, \textup{#3(#1)#4--#5(#2)#6}}{, and \textup{#3(#1)#4--#5(#2)#6}}

% But spell it out at the beginning of a sentence.
\Crefformat{equation}{#2Equation~\textup{(#1)}#3}
\Crefrangeformat{equation}{Equations~\textup{#3(#1)#4--#5(#2)#6}}
\Crefmultiformat{equation}{Equations~\textup{#2(#1)#3}}{ and \textup{#2(#1)#3}}
{, \textup{#2(#1)#3}}{, and \textup{#2(#1)#3}}
\Crefrangemultiformat{equation}{Equations~\textup{#3(#1)#4--#5(#2)#6}}%
{ and \textup{#3(#1)#4--#5(#2)#6}}{, \textup{#3(#1)#4--#5(#2)#6}}{, and \textup{#3(#1)#4--#5(#2)#6}}

% Make number non-italic in any environment.
\crefdefaultlabelformat{#2\textup{#1}#3}

% My name
\author{Jaden Wang}



\begin{document}
\subsection{First-order conditions}

~\begin{thm}
	If $ f: \rr^{n} \to \rr$ is differentiable on dom(f) and if dom(f) is open and convex, then $ f$ is convex iff for all  $ x,y \in \dom(f)$,
	\[
		f(y) \geq f(x) + \langle \nabla f(x),y-x \rangle
	.\] 
\end{thm}
\begin{note}
	This is the 1st order Taylor approximation (tangent line). The line is supporting the epigraph of $ f$.
\end{note}
~\begin{figure}[H]
	\centering
	\includegraphics[width=\textwidth]{./figures/cvx_tan.png}
\end{figure}
\begin{thm}
	Under the same assumption, $ f$ is convex iff $\nabla  f$ is monotone. That is, for all $ x,y \in \dom(x)$,
	\[
		\langle x-y, \nabla f(x) - \nabla f(y) \rangle \geq 0
	.\] 
\end{thm}
\begin{intuition}
	Recall in 1D, $ f$ is convex if slope is non-decreasing. That is, if $ x-y\geq 0$, then  $ f'(x) - f'(y) \geq 0$ and if $ x-y \leq 0$ then  $ f'(x)-f'(y)\leq 0$. A concise way to express that is $ (x-y)(f'(x) - f'(y)) \geq 0$. Here we generalize this to higher dimensions.
\end{intuition}

\begin{thm}[2nd-order condition]
	$ f: \rr^{n} \to \rr$. If the Hessian $ \nabla^2 f(x)$ exists for all $ x \in \dom(f)$, then
	\begin{enumerate}[label=\alph*)]
		\item $ f$ is convex iff  $ \nabla ^2 f(x) \succeq 0 \ \forall \ x \in \dom(f)$.
		\item $ f$ is  $ \mu$-strongly convex ( w.r.t. $ \norm{ \cdot }_2 $ ) iff $ \nabla ^2 f(x) \succeq \mu I$.
	\end{enumerate}
	If $ \nabla ^2 f(x) \succ 0$, then $ f$ is \allbold{strictly convex}. 
\end{thm}
\begin{remark}
$ f$ can be convex but  $ \nabla f, \nabla ^2 f$ need not exist!
\end{remark}
What if $ f$ isn't differentiable?

 \begin{defn}[subdifferential]
	 Let $ f: \rr^{n} \to (-\infty, \infty]$ be proper, then we define the \allbold{subdifferential} of $ f$ at $ x$ to be
	 \[
		 \partial f(x) = \{d \in \rr^{n}: \ \forall \ y \in \rr^{n}, f(y) \geq f(x) + \langle d,y-x \rangle\} 
	 .\] 
\end{defn}
\begin{note}
$ d$ here is called a  \allbold{subgradient}. 
\end{note}
\begin{thm}
If $ f$ is proper and convex then
 \[
	 x \in \ri(\dom(f)) \implies \partial f(x) \neq \O
.\] 
\end{thm}
\begin{note}
The proof is related to separating/supporting hyperplanes.
\end{note}
\begin{prop}
	$ \partial f(x) $ is a singleton  iff  $ f$ is differentiable at  $ x$.
\end{prop}

\begin{eg}
	$ f(x) = |x|$. Then if  $ x\neq 0$,  $ f'(x) = \sgn(x)$ and  $ \partial f(x) = \{f'(x)\} $. If $ x=0$,  $ f'(0)$ DNE. But $ \partial f(0) = [-1,1] $.
\end{eg}

\begin{thm}[Fermat's Rule]
If $ f$ is a proper function, then
 \[
	 \argmin_{x} f(x) = \{x: 0 \in \partial f(x)\} 
.\] 
\end{thm}
\begin{proof}
This just means that we can plug $ 0$ into the definition of subdifferential and get
 \[
	 f(y) \geq f(x) + \langle 0, y-x \rangle = f(x) \ \forall \ y
.\] 
This clearly shows that $ x$ is a global minimizer. 
\end{proof}

\begin{note}
This generalizes the calculus idea of critical points for smooth functions.
\end{note}

\begin{remark}
	Subdifferentials are a global notion (for all $ y$) whereas gradients are a local notion. How do we reconcile that subdifferential can be the gradient? The answer is that the global property of convexity links the two.
\end{remark}
\begin{remark}
So all we need to do is to invert $ \partial f$. That is,
 \[
	 \argmin f(x) = \partial f^{-1}
.\]
In fact, this is usually not practical or even possible especially for interesting problems. It may be possible for subproblems.
\end{remark}
\begin{defn}[normal cone]
The \allbold{normal cone} to a set $ C$ at point  $ x$ is
\begin{equation*}
	N_C(x)=
\begin{cases}
	\{d: \langle d,y-x \rangle \leq 0 \ \forall \ y \in C\} & \text{ if } x \in C\\
	\O & \text{ if } x \not\in C 
\end{cases}
\end{equation*}
\end{defn}
\begin{eg}
Let $ C \neq \O$ be convex, so $ I_C$ is a proper convex function. Then  $ \partial I_C = N_C$.
\end{eg}
\begin{eg}
	$ x \in \inte C \implies N_C(x) = \{0\} $. Why? WLOG, shift $ C$ so  $ x=0$. If  $ \langle d,y \rangle\leq 0 \ \forall \ y \in C$. Then $ x \in \inte C \implies$ we can choose $y = \epsilon d \in C$ for sufficiently small $ \epsilon >0$. Then $ \epsilon \norm{ d}^2 \leq 0 \implies d=0 $.
\end{eg}
\begin{eg}
	$ x \in \partial C$ (the boundary). We want $ d$  s.t. $ \langle d,y \rangle\leq 0 \ \forall \ y \in C$. Geometrically this means we want the angle between $ d,y$ to be perpendicular or obtuse. If the boundary is smooth, since  $ d$ needs to be at least perpendicular to any  $ y$ immediately to the left and right of  $ x$, it must be the normal ray of the tangent plane.
\end{eg}

\begin{eg}
	~\begin{figure}[H]
		\centering
		\includegraphics[width=0.8\textwidth]{./figures/normal_cone.png}
		\caption{The normal cone at non-smooth boundary looks indeed like a cone.}
	\end{figure}
\end{eg}
\begin{remark}
An equivalent definition of normal cone is the set of all vectors that define a supporting hyperplane to $ C$, passing through  $ x$.
\end{remark}

\begin{eg}
If $ C$ is a vector space, since $ C$ is closed under inverses, if we use  $ -y$ in addition to $ y$ in the definition we will get an equality which implies orthogonality. Hence
 \begin{equation*}
	 N_C(x)=
\begin{cases}
	C^{\perp} & x \in C\\
	\O & x \not\in C
\end{cases}
\end{equation*}
\end{eg}

\begin{prop}[6.47 BC17]
	If $ C \neq \O$ is closed and convex, then $ x=P_C(y)$ iff  $ y-x \in N_C(x)$, where $ P_C(y)$ denotes the orthogonal projection of  $ y$ onto  $ C$.
\end{prop}
\end{document}
