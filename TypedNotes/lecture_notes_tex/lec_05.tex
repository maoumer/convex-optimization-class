\documentclass[class=article,crop=false]{standalone} 
%Fall 2020
% Some basic packages
\usepackage{standalone}[subpreambles=true]
\usepackage[utf8]{inputenc}
\usepackage[T1]{fontenc}
\usepackage{textcomp}
\usepackage[english]{babel}
\usepackage{url}
\usepackage{graphicx}
\usepackage{float}
\usepackage{enumitem}
\usepackage{lmodern}
\usepackage{hyperref}
\usepackage[usenames,svgnames,dvipsnames]{xcolor}


\pdfminorversion=7

% Don't indent paragraphs, leave some space between them
\usepackage{parskip}

% Hide page number when page is empty
\usepackage{emptypage}
\usepackage{subcaption}
\usepackage{multicol}
\usepackage[dvipsnames]{xcolor}
\usepackage[b]{esvect}

% Math stuff
\usepackage{amsmath, amsfonts, mathtools, amsthm, amssymb}
\usepackage{bbm}

% Fancy script capitals
\usepackage{mathrsfs}
\usepackage{cancel}
% Bold math
\usepackage{bm}
% Some shortcuts
\newcommand{\rr}{\ensuremath{\mathbb{R}}}
\newcommand{\zz}{\ensuremath{\mathbb{Z}}}
\newcommand{\qq}{\ensuremath{\mathbb{Q}}}
\newcommand{\nn}{\ensuremath{\mathbb{N}}}
\newcommand{\ff}{\ensuremath{\mathbb{F}}}
\newcommand{\cc}{\ensuremath{\mathbb{C}}}
\newcommand{\ee}{\ensuremath{\mathbb{E}}}
\renewcommand\O{\ensuremath{\emptyset}}
\newcommand{\norm}[1]{{\left\lVert{#1}\right\rVert}}
\newcommand{\ve}[1]{{\mathbf{#1}}}
\newcommand\allbold[1]{{\boldmath\textbf{#1}}}
\DeclareMathOperator{\lcm}{lcm}
\DeclareMathOperator{\im}{im}
\DeclareMathOperator{\coim}{coim}
\DeclareMathOperator{\dom}{dom}
\DeclareMathOperator{\tr}{tr}
\DeclareMathOperator{\rank}{rank}
\DeclareMathOperator*{\var}{Var}
\DeclareMathOperator*{\ev}{E}
\DeclareMathOperator{\sinc}{sinc}
\DeclareMathOperator{\dg}{deg}
\DeclareMathOperator{\aff}{aff}
\DeclareMathOperator{\conv}{conv}
\DeclareMathOperator{\epi}{epi}
\DeclareMathOperator{\inte}{int}
\DeclareMathOperator{\ri}{ri}
\DeclareMathOperator*{\argmin}{argmin}
\DeclareMathOperator*{\argmax}{argmax}
\DeclareMathOperator{\graph}{graph}
\DeclareMathOperator{\sgn}{sgn}
\DeclareMathOperator*{\Rep}{Rep}
\DeclareMathOperator{\Proj}{Proj}
\DeclareMathOperator{\prox}{prox}
\DeclareMathOperator{\mat}{mat}
\let\vec\relax
\DeclareMathOperator{\vec}{vec}
\let\Re\relax
\DeclareMathOperator{\Re}{Re}
\let\Im\relax
\DeclareMathOperator{\Im}{Im}
% Put x \to \infty below \lim
\let\svlim\lim\def\lim{\svlim\limits}

%wide hat
\usepackage{scalerel,stackengine}
\stackMath
\newcommand*\wh[1]{%
\savestack{\tmpbox}{\stretchto{%
  \scaleto{%
    \scalerel*[\widthof{\ensuremath{#1}}]{\kern-.6pt\bigwedge\kern-.6pt}%
    {\rule[-\textheight/2]{1ex}{\textheight}}%WIDTH-LIMITED BIG WEDGE
  }{\textheight}% 
}{0.5ex}}%
\stackon[1pt]{#1}{\tmpbox}%
}
\parskip 1ex

%Make implies and impliedby shorter
\let\implies\Rightarrow
\let\impliedby\Leftarrow
\let\iff\Leftrightarrow
\let\epsilon\varepsilon

% Add \contra symbol to denote contradiction
\usepackage{stmaryrd} % for \lightning
\newcommand\contra{\scalebox{1.5}{$\lightning$}}

% \let\phi\varphi

% Command for short corrections
% Usage: 1+1=\correct{3}{2}

\definecolor{correct}{HTML}{009900}
\newcommand\correct[2]{\ensuremath{\:}{\color{red}{#1}}\ensuremath{\to }{\color{correct}{#2}}\ensuremath{\:}}
\newcommand\green[1]{{\color{correct}{#1}}}

% horizontal rule
\newcommand\hr{
    \noindent\rule[0.5ex]{\linewidth}{0.5pt}
}

% hide parts
\newcommand\hide[1]{}

% si unitx
\usepackage{siunitx}
\sisetup{locale = FR}

%allows pmatrix to stretch
\makeatletter
\renewcommand*\env@matrix[1][\arraystretch]{%
  \edef\arraystretch{#1}%
  \hskip -\arraycolsep
  \let\@ifnextchar\new@ifnextchar
  \array{*\c@MaxMatrixCols c}}
\makeatother

\renewcommand{\arraystretch}{0.8}

% Environments
\makeatother
% For box around Definition, Theorem, \ldots
%%fakesection Theorems
\usepackage{thmtools}
\usepackage[framemethod=TikZ]{mdframed}

\theoremstyle{definition}
\mdfdefinestyle{mdbluebox}{%
	roundcorner = 10pt,
	linewidth=1pt,
	skipabove=12pt,
	innerbottommargin=9pt,
	skipbelow=2pt,
	nobreak=true,
	linecolor=blue,
	backgroundcolor=TealBlue!5,
}
\declaretheoremstyle[
	headfont=\sffamily\bfseries\color{MidnightBlue},
	mdframed={style=mdbluebox},
	headpunct={\\[3pt]},
	postheadspace={0pt}
]{thmbluebox}

\mdfdefinestyle{mdredbox}{%
	linewidth=0.5pt,
	skipabove=12pt,
	frametitleaboveskip=5pt,
	frametitlebelowskip=0pt,
	skipbelow=2pt,
	frametitlefont=\bfseries,
	innertopmargin=4pt,
	innerbottommargin=8pt,
	nobreak=false,
	linecolor=RawSienna,
	backgroundcolor=Salmon!5,
}
\declaretheoremstyle[
	headfont=\bfseries\color{RawSienna},
	mdframed={style=mdredbox},
	headpunct={\\[3pt]},
	postheadspace={0pt},
]{thmredbox}

\declaretheorem[%
style=thmbluebox,name=Theorem,numberwithin=section]{thm}
\declaretheorem[style=thmbluebox,name=Lemma,sibling=thm]{lem}
\declaretheorem[style=thmbluebox,name=Proposition,sibling=thm]{prop}
\declaretheorem[style=thmbluebox,name=Corollary,sibling=thm]{coro}
\declaretheorem[style=thmredbox,name=Example,sibling=thm]{eg}

\mdfdefinestyle{mdgreenbox}{%
	roundcorner = 10pt,
	linewidth=1pt,
	skipabove=12pt,
	innerbottommargin=9pt,
	skipbelow=2pt,
	nobreak=true,
	linecolor=ForestGreen,
	backgroundcolor=ForestGreen!5,
}

\declaretheoremstyle[
	headfont=\bfseries\sffamily\color{ForestGreen!70!black},
	bodyfont=\normalfont,
	spaceabove=2pt,
	spacebelow=1pt,
	mdframed={style=mdgreenbox},
	headpunct={ --- },
]{thmgreenbox}

\declaretheorem[style=thmgreenbox,name=Definition,sibling=thm]{defn}

\mdfdefinestyle{mdgreenboxsq}{%
	linewidth=1pt,
	skipabove=12pt,
	innerbottommargin=9pt,
	skipbelow=2pt,
	nobreak=true,
	linecolor=ForestGreen,
	backgroundcolor=ForestGreen!5,
}
\declaretheoremstyle[
	headfont=\bfseries\sffamily\color{ForestGreen!70!black},
	bodyfont=\normalfont,
	spaceabove=2pt,
	spacebelow=1pt,
	mdframed={style=mdgreenboxsq},
	headpunct={},
]{thmgreenboxsq}
\declaretheoremstyle[
	headfont=\bfseries\sffamily\color{ForestGreen!70!black},
	bodyfont=\normalfont,
	spaceabove=2pt,
	spacebelow=1pt,
	mdframed={style=mdgreenboxsq},
	headpunct={},
]{thmgreenboxsq*}

\mdfdefinestyle{mdblackbox}{%
	skipabove=8pt,
	linewidth=3pt,
	rightline=false,
	leftline=true,
	topline=false,
	bottomline=false,
	linecolor=black,
	backgroundcolor=RedViolet!5!gray!5,
}
\declaretheoremstyle[
	headfont=\bfseries,
	bodyfont=\normalfont\small,
	spaceabove=0pt,
	spacebelow=0pt,
	mdframed={style=mdblackbox}
]{thmblackbox}

\theoremstyle{plain}
\declaretheorem[name=Question,sibling=thm,style=thmblackbox]{ques}
\declaretheorem[name=Remark,sibling=thm,style=thmgreenboxsq]{remark}
\declaretheorem[name=Remark,sibling=thm,style=thmgreenboxsq*]{remark*}

\theoremstyle{definition}
\newtheorem{claim}[thm]{Claim}
\theoremstyle{remark}
\newtheorem*{case}{Case}
\newtheorem*{notation}{Notation}
\newtheorem*{note}{Note}
\newtheorem*{motivation}{Motivation}
\newtheorem*{intuition}{Intuition}

% Make section starts with 1 for report type
%\renewcommand\thesection{\arabic{section}}

% End example and intermezzo environments with a small diamond (just like proof
% environments end with a small square)
\usepackage{etoolbox}
\AtEndEnvironment{vb}{\null\hfill$\diamond$}%
\AtEndEnvironment{intermezzo}{\null\hfill$\diamond$}%
% \AtEndEnvironment{opmerking}{\null\hfill$\diamond$}%

% Fix some spacing
% http://tex.stackexchange.com/questions/22119/how-can-i-change-the-spacing-before-theorems-with-amsthm
\makeatletter
\def\thm@space@setup{%
  \thm@preskip=\parskip \thm@postskip=0pt
}

% Fix some stuff
% %http://tex.stackexchange.com/questions/76273/multiple-pdfs-with-page-group-included-in-a-single-page-warning
\pdfsuppresswarningpagegroup=1

\renewcommand{\baselinestretch}{1.5}
\RequirePackage{hyperref}[6.83]
\hypersetup{
  colorlinks=false,
  frenchlinks=false,
  pdfborder={0 0 0},
  naturalnames=false,
  hypertexnames=false,
  breaklinks
}
\urlstyle{same}

\usepackage{graphics}
\usepackage{epstopdf}

%%
%% Add support for color in order to color the hyperlinks.
%% 
\hypersetup{
  colorlinks = true,
  allcolors = siaminlinkcolor,
  urlcolor = siamexlinkcolor,
}
%%fakesection Links
\hypersetup{
    colorlinks,
    linkcolor={red!50!black},
    citecolor={green!50!black},
    urlcolor={blue!80!black}
}
%customization of cleveref
\RequirePackage[capitalize,nameinlink]{cleveref}[0.19]

% Per SIAM Style Manual, "section" should be lowercase
\crefname{section}{section}{sections}
\crefname{subsection}{subsection}{subsections}
\Crefname{section}{Section}{Sections}
\Crefname{subsection}{Subsection}{Subsections}

% Per SIAM Style Manual, "Figure" should be spelled out in references
\Crefname{figure}{Figure}{Figures}

% Per SIAM Style Manual, don't say equation in front on an equation.
\crefformat{equation}{\textup{#2(#1)#3}}
\crefrangeformat{equation}{\textup{#3(#1)#4--#5(#2)#6}}
\crefmultiformat{equation}{\textup{#2(#1)#3}}{ and \textup{#2(#1)#3}}
{, \textup{#2(#1)#3}}{, and \textup{#2(#1)#3}}
\crefrangemultiformat{equation}{\textup{#3(#1)#4--#5(#2)#6}}%
{ and \textup{#3(#1)#4--#5(#2)#6}}{, \textup{#3(#1)#4--#5(#2)#6}}{, and \textup{#3(#1)#4--#5(#2)#6}}

% But spell it out at the beginning of a sentence.
\Crefformat{equation}{#2Equation~\textup{(#1)}#3}
\Crefrangeformat{equation}{Equations~\textup{#3(#1)#4--#5(#2)#6}}
\Crefmultiformat{equation}{Equations~\textup{#2(#1)#3}}{ and \textup{#2(#1)#3}}
{, \textup{#2(#1)#3}}{, and \textup{#2(#1)#3}}
\Crefrangemultiformat{equation}{Equations~\textup{#3(#1)#4--#5(#2)#6}}%
{ and \textup{#3(#1)#4--#5(#2)#6}}{, \textup{#3(#1)#4--#5(#2)#6}}{, and \textup{#3(#1)#4--#5(#2)#6}}

% Make number non-italic in any environment.
\crefdefaultlabelformat{#2\textup{#1}#3}

% My name
\author{Jaden Wang}



\begin{document}

\subsection{separating and supporting hyperplanes}
~\begin{thm}[separating hyperplane]
Let $ C,D$ be convex, non-intersecting sets in  $ \rr^{n}$, then there exists $ a \in \rr^{n} \setminus \{0\} $ and $ \mu \in \rr$ s.t. 
\begin{align*}
	a^{T}x \leq \mu \ \forall \ x \in C\\
	a^{T} x \geq \mu \ \forall \ x \in D
\end{align*}
\end{thm}
\begin{note}
This reads as there exists a hyperplane that separates the two convex sets. It is clearly not true if the sets aren't convex. $ a$ is the normal to the hyperplane.
\end{note}

\begin{defn}[Chebyshev set]
A set $ S$ is a  \allbold{Chebyshev set} if for all $ x_0$, there exists a unique $ x \in S$ s.t. 
\[
x = \argmin_{y \in S} \norm{ y-x_0} 
.\] 
\end{defn}
\begin{note}
This reads as there exists a unique best approximation point in the set $ S$ for any $ x_0$.
\end{note}
\begin{eg}
Open unit ball isn't Chebyshev because it doesn't reach infimum.
\end{eg}
\begin{eg}
A nonconvex set isn't Chebyshev because there exists an $ x_0$ where we have at least two best approximation points.
\end{eg}
\begin{thm}
Any nonempty, closed, convex set in a Hilbert space is Chebyshev.
\end{thm}

\begin{thm}[supporting hyperplanes]
~\begin{enumerate}[label=(\roman*)]
	\item If $ C$ is convex, closed and  $ D = \{x_0\}, x_0 \not\in C $, then there exists $ a \in \rr^{n}$ s.t. $ a^{T}x< a^{T}x_0 \ \forall \ x \in C$.
	\item Same but $ C$ needs not be closed,  $ x_0 \not\in  \overline{C}$.
	\item as in (ii) but allow $ x_0 \in \overline{C}\setminus C$.
\end{enumerate}
\end{thm}
\begin{proof}
	(i): WLOG let $ x_0 = 0$ (since we can always translate $ C$). $ C$ is Chebyshev so let  $ y$ be the unique closest point to  $ 0$, and define  $ a=-y$ (normal of the hyperplane). We wish to show that $ a^{T}x< a^{T} x_0 =0 \ \forall \ x \in C$. That is, $ y^{T} x>0 \ \forall \ x \in C$.

	Given $ x \in C$, $ y + \epsilon (x-y) \in C$ by convexity. Since $ y$ is the best approximation point,
	\begin{align*}
		\norm{ y}^2 &\leq \norm{ y + \epsilon(x-y)}^2 \\
		&= \norm{ y}^2 + 2 \epsilon\langle y,x-y \rangle+ \epsilon^2 \norm{ x-y}^2   \\
		0&= 2 \langle y,x \rangle - 2 \langle y,y \rangle + \epsilon \norm{ x-y}^2  \\
		\langle y,x \rangle &\geq \norm{ y}^2 -\frac{ \epsilon}{2} \norm{ x-y}^2 
	\end{align*}
	Take $ \epsilon \to 0$, since $ y \neq 0 \implies \norm{ y}>0 $, we obtain $ y^{T}x>0$ as required.
\end{proof}

\begin{remark}
	This is related to \allbold{Theorems of Alternatives}. Generally, they are stated as the following:

Either $ A$ is true,  $ B$ is false, but not both.
\end{remark}
\begin{eg}[Fredhold alternative,finite-dim]

Either $ \{x: Ax=b\} $ is empty, or $ \{\lambda: A^{T} \lambda =0, \lambda^{T}b \neq 0\} $ is non-empty, but not both.

Why do we care? To prove that there is a solution to $ Ax=b$. We can simply find a solution  $ x$. This is a "certificate". But if professor asks you to prove there isn't a solution to  $ Ax=b$, we can try to show that  $ A$ is singular, but if  $ b=0$ even singular  $ A$ works. Another way is to find a "certificate"  $ \lambda$. This is the first task of duality.
\end{eg}


\begin{eg}[Farkas Lemma]
Either $ \{Ax=b,x\geq 0\} $ is non-empty, or $ \{\lambda: A^{T} \lambda\geq 0, \lambda^{T}b<0\} $ is non-empty, but not both.
\end{eg}
\begin{thm}[Thereom of Alternatives for strict linear inequalities]
The following statements are equivalent:
\begin{enumerate}[label=(\roman*)]
	\item The set $ \{x: Ax<b\} $ is empty.
	\item The sets $ C = \{b-Ax : x \in \rr^{n}\} $ and $ D = \rr_{++}^{m}$ do not intersect.
	\item The hyperplane separation theorem and its converse hold. That is,
		\[
			\ \exists \ \lambda \geq 0 ( \lambda \neq 0) \text{ s.t. } A^{T} \lambda = 0, \lambda ^{T} b \leq 0 
		.\] 
\end{enumerate}
\end{thm}
\begin{intuition}
	(ii) is just rephrasing (i). No intersection from (ii) can then be established by finding something that separates $ C,D$ in (iii).
\end{intuition}
\begin{proof}(converse of hyperplane separation)

	(iii) $ \implies$ (i): suppose such $ \lambda$ exists, and for contradiction, assume there exists $ x$  s.t. $ Ax < b$. Then since $ \lambda \geq 0$,
\[
	0= (A^{T} \lambda)^{T} x = \lambda^{T} Ax < \lambda ^{T} b
.\] 
So we obtain $ 0< \lambda^{T} b \leq 0$, a contradiction.

(i) $ \implies$ (iii): By the separation theorem, we know there exists $ \lambda \neq 0$ s.t. 
\begin{align*}
	\lambda^{T}(b-Ax) &\leq \mu, x \in \rr^{n}\\
	\lambda^{T} y &\geq \mu, y \in \rr_{++}^{n}
\end{align*}
It follows from the first condition that $ \lambda^{T}Ax =0$ because otherwise we can just choose a large negative $ x$ to exceed $ \mu$ and get contradiction. Since this is true for all $ x$, it must be that  $ \lambda^{T} A = A^{T} \lambda =0$. From the second condition we have $ \lambda \geq 0$, because otherwise if $ \lambda_i<0$, we can choose $ y_i \to \infty$ to get contradiction. Moreover, we need $ \mu \leq 0$ since if $ \mu>0$, we can take all components of $ y $ to $ 0^{+} $, so $ \lambda^{T} y \to 0^{+} $. Then $ \lambda^{T}(b-A^{T}x)\leq \mu \leq 0$ implies that $ \lambda^{T} b \leq 0$.

Taken together, we have $ \lambda \geq 0, \lambda \neq 0$, $ A^{T} \lambda =0,$ and $ \lambda^{T}b \leq 0$.
\end{proof}
\newpage
\end{document}
