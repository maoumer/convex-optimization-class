\documentclass[class=article,crop=false]{standalone} 
%Fall 2020
% Some basic packages
\usepackage{standalone}[subpreambles=true]
\usepackage[utf8]{inputenc}
\usepackage[T1]{fontenc}
\usepackage{textcomp}
\usepackage[english]{babel}
\usepackage{url}
\usepackage{graphicx}
\usepackage{float}
\usepackage{enumitem}
\usepackage{lmodern}
\usepackage{hyperref}
\usepackage[usenames,svgnames,dvipsnames]{xcolor}


\pdfminorversion=7

% Don't indent paragraphs, leave some space between them
\usepackage{parskip}

% Hide page number when page is empty
\usepackage{emptypage}
\usepackage{subcaption}
\usepackage{multicol}
\usepackage[dvipsnames]{xcolor}
\usepackage[b]{esvect}

% Math stuff
\usepackage{amsmath, amsfonts, mathtools, amsthm, amssymb}
\usepackage{bbm}

% Fancy script capitals
\usepackage{mathrsfs}
\usepackage{cancel}
% Bold math
\usepackage{bm}
% Some shortcuts
\newcommand{\rr}{\ensuremath{\mathbb{R}}}
\newcommand{\zz}{\ensuremath{\mathbb{Z}}}
\newcommand{\qq}{\ensuremath{\mathbb{Q}}}
\newcommand{\nn}{\ensuremath{\mathbb{N}}}
\newcommand{\ff}{\ensuremath{\mathbb{F}}}
\newcommand{\cc}{\ensuremath{\mathbb{C}}}
\newcommand{\ee}{\ensuremath{\mathbb{E}}}
\renewcommand\O{\ensuremath{\emptyset}}
\newcommand{\norm}[1]{{\left\lVert{#1}\right\rVert}}
\newcommand{\ve}[1]{{\mathbf{#1}}}
\newcommand\allbold[1]{{\boldmath\textbf{#1}}}
\DeclareMathOperator{\lcm}{lcm}
\DeclareMathOperator{\im}{im}
\DeclareMathOperator{\coim}{coim}
\DeclareMathOperator{\dom}{dom}
\DeclareMathOperator{\tr}{tr}
\DeclareMathOperator{\rank}{rank}
\DeclareMathOperator*{\var}{Var}
\DeclareMathOperator*{\ev}{E}
\DeclareMathOperator{\sinc}{sinc}
\DeclareMathOperator{\dg}{deg}
\DeclareMathOperator{\aff}{aff}
\DeclareMathOperator{\conv}{conv}
\DeclareMathOperator{\epi}{epi}
\DeclareMathOperator{\inte}{int}
\DeclareMathOperator{\ri}{ri}
\DeclareMathOperator*{\argmin}{argmin}
\DeclareMathOperator*{\argmax}{argmax}
\DeclareMathOperator{\graph}{graph}
\DeclareMathOperator{\sgn}{sgn}
\DeclareMathOperator*{\Rep}{Rep}
\DeclareMathOperator{\Proj}{Proj}
\DeclareMathOperator{\prox}{prox}
\DeclareMathOperator{\mat}{mat}
\let\vec\relax
\DeclareMathOperator{\vec}{vec}
\let\Re\relax
\DeclareMathOperator{\Re}{Re}
\let\Im\relax
\DeclareMathOperator{\Im}{Im}
% Put x \to \infty below \lim
\let\svlim\lim\def\lim{\svlim\limits}

%wide hat
\usepackage{scalerel,stackengine}
\stackMath
\newcommand*\wh[1]{%
\savestack{\tmpbox}{\stretchto{%
  \scaleto{%
    \scalerel*[\widthof{\ensuremath{#1}}]{\kern-.6pt\bigwedge\kern-.6pt}%
    {\rule[-\textheight/2]{1ex}{\textheight}}%WIDTH-LIMITED BIG WEDGE
  }{\textheight}% 
}{0.5ex}}%
\stackon[1pt]{#1}{\tmpbox}%
}
\parskip 1ex

%Make implies and impliedby shorter
\let\implies\Rightarrow
\let\impliedby\Leftarrow
\let\iff\Leftrightarrow
\let\epsilon\varepsilon

% Add \contra symbol to denote contradiction
\usepackage{stmaryrd} % for \lightning
\newcommand\contra{\scalebox{1.5}{$\lightning$}}

% \let\phi\varphi

% Command for short corrections
% Usage: 1+1=\correct{3}{2}

\definecolor{correct}{HTML}{009900}
\newcommand\correct[2]{\ensuremath{\:}{\color{red}{#1}}\ensuremath{\to }{\color{correct}{#2}}\ensuremath{\:}}
\newcommand\green[1]{{\color{correct}{#1}}}

% horizontal rule
\newcommand\hr{
    \noindent\rule[0.5ex]{\linewidth}{0.5pt}
}

% hide parts
\newcommand\hide[1]{}

% si unitx
\usepackage{siunitx}
\sisetup{locale = FR}

%allows pmatrix to stretch
\makeatletter
\renewcommand*\env@matrix[1][\arraystretch]{%
  \edef\arraystretch{#1}%
  \hskip -\arraycolsep
  \let\@ifnextchar\new@ifnextchar
  \array{*\c@MaxMatrixCols c}}
\makeatother

\renewcommand{\arraystretch}{0.8}

% Environments
\makeatother
% For box around Definition, Theorem, \ldots
%%fakesection Theorems
\usepackage{thmtools}
\usepackage[framemethod=TikZ]{mdframed}

\theoremstyle{definition}
\mdfdefinestyle{mdbluebox}{%
	roundcorner = 10pt,
	linewidth=1pt,
	skipabove=12pt,
	innerbottommargin=9pt,
	skipbelow=2pt,
	nobreak=true,
	linecolor=blue,
	backgroundcolor=TealBlue!5,
}
\declaretheoremstyle[
	headfont=\sffamily\bfseries\color{MidnightBlue},
	mdframed={style=mdbluebox},
	headpunct={\\[3pt]},
	postheadspace={0pt}
]{thmbluebox}

\mdfdefinestyle{mdredbox}{%
	linewidth=0.5pt,
	skipabove=12pt,
	frametitleaboveskip=5pt,
	frametitlebelowskip=0pt,
	skipbelow=2pt,
	frametitlefont=\bfseries,
	innertopmargin=4pt,
	innerbottommargin=8pt,
	nobreak=false,
	linecolor=RawSienna,
	backgroundcolor=Salmon!5,
}
\declaretheoremstyle[
	headfont=\bfseries\color{RawSienna},
	mdframed={style=mdredbox},
	headpunct={\\[3pt]},
	postheadspace={0pt},
]{thmredbox}

\declaretheorem[%
style=thmbluebox,name=Theorem,numberwithin=section]{thm}
\declaretheorem[style=thmbluebox,name=Lemma,sibling=thm]{lem}
\declaretheorem[style=thmbluebox,name=Proposition,sibling=thm]{prop}
\declaretheorem[style=thmbluebox,name=Corollary,sibling=thm]{coro}
\declaretheorem[style=thmredbox,name=Example,sibling=thm]{eg}

\mdfdefinestyle{mdgreenbox}{%
	roundcorner = 10pt,
	linewidth=1pt,
	skipabove=12pt,
	innerbottommargin=9pt,
	skipbelow=2pt,
	nobreak=true,
	linecolor=ForestGreen,
	backgroundcolor=ForestGreen!5,
}

\declaretheoremstyle[
	headfont=\bfseries\sffamily\color{ForestGreen!70!black},
	bodyfont=\normalfont,
	spaceabove=2pt,
	spacebelow=1pt,
	mdframed={style=mdgreenbox},
	headpunct={ --- },
]{thmgreenbox}

\declaretheorem[style=thmgreenbox,name=Definition,sibling=thm]{defn}

\mdfdefinestyle{mdgreenboxsq}{%
	linewidth=1pt,
	skipabove=12pt,
	innerbottommargin=9pt,
	skipbelow=2pt,
	nobreak=true,
	linecolor=ForestGreen,
	backgroundcolor=ForestGreen!5,
}
\declaretheoremstyle[
	headfont=\bfseries\sffamily\color{ForestGreen!70!black},
	bodyfont=\normalfont,
	spaceabove=2pt,
	spacebelow=1pt,
	mdframed={style=mdgreenboxsq},
	headpunct={},
]{thmgreenboxsq}
\declaretheoremstyle[
	headfont=\bfseries\sffamily\color{ForestGreen!70!black},
	bodyfont=\normalfont,
	spaceabove=2pt,
	spacebelow=1pt,
	mdframed={style=mdgreenboxsq},
	headpunct={},
]{thmgreenboxsq*}

\mdfdefinestyle{mdblackbox}{%
	skipabove=8pt,
	linewidth=3pt,
	rightline=false,
	leftline=true,
	topline=false,
	bottomline=false,
	linecolor=black,
	backgroundcolor=RedViolet!5!gray!5,
}
\declaretheoremstyle[
	headfont=\bfseries,
	bodyfont=\normalfont\small,
	spaceabove=0pt,
	spacebelow=0pt,
	mdframed={style=mdblackbox}
]{thmblackbox}

\theoremstyle{plain}
\declaretheorem[name=Question,sibling=thm,style=thmblackbox]{ques}
\declaretheorem[name=Remark,sibling=thm,style=thmgreenboxsq]{remark}
\declaretheorem[name=Remark,sibling=thm,style=thmgreenboxsq*]{remark*}

\theoremstyle{definition}
\newtheorem{claim}[thm]{Claim}
\theoremstyle{remark}
\newtheorem*{case}{Case}
\newtheorem*{notation}{Notation}
\newtheorem*{note}{Note}
\newtheorem*{motivation}{Motivation}
\newtheorem*{intuition}{Intuition}

% Make section starts with 1 for report type
%\renewcommand\thesection{\arabic{section}}

% End example and intermezzo environments with a small diamond (just like proof
% environments end with a small square)
\usepackage{etoolbox}
\AtEndEnvironment{vb}{\null\hfill$\diamond$}%
\AtEndEnvironment{intermezzo}{\null\hfill$\diamond$}%
% \AtEndEnvironment{opmerking}{\null\hfill$\diamond$}%

% Fix some spacing
% http://tex.stackexchange.com/questions/22119/how-can-i-change-the-spacing-before-theorems-with-amsthm
\makeatletter
\def\thm@space@setup{%
  \thm@preskip=\parskip \thm@postskip=0pt
}

% Fix some stuff
% %http://tex.stackexchange.com/questions/76273/multiple-pdfs-with-page-group-included-in-a-single-page-warning
\pdfsuppresswarningpagegroup=1

\renewcommand{\baselinestretch}{1.5}
\RequirePackage{hyperref}[6.83]
\hypersetup{
  colorlinks=false,
  frenchlinks=false,
  pdfborder={0 0 0},
  naturalnames=false,
  hypertexnames=false,
  breaklinks
}
\urlstyle{same}

\usepackage{graphics}
\usepackage{epstopdf}

%%
%% Add support for color in order to color the hyperlinks.
%% 
\hypersetup{
  colorlinks = true,
  allcolors = siaminlinkcolor,
  urlcolor = siamexlinkcolor,
}
%%fakesection Links
\hypersetup{
    colorlinks,
    linkcolor={red!50!black},
    citecolor={green!50!black},
    urlcolor={blue!80!black}
}
%customization of cleveref
\RequirePackage[capitalize,nameinlink]{cleveref}[0.19]

% Per SIAM Style Manual, "section" should be lowercase
\crefname{section}{section}{sections}
\crefname{subsection}{subsection}{subsections}
\Crefname{section}{Section}{Sections}
\Crefname{subsection}{Subsection}{Subsections}

% Per SIAM Style Manual, "Figure" should be spelled out in references
\Crefname{figure}{Figure}{Figures}

% Per SIAM Style Manual, don't say equation in front on an equation.
\crefformat{equation}{\textup{#2(#1)#3}}
\crefrangeformat{equation}{\textup{#3(#1)#4--#5(#2)#6}}
\crefmultiformat{equation}{\textup{#2(#1)#3}}{ and \textup{#2(#1)#3}}
{, \textup{#2(#1)#3}}{, and \textup{#2(#1)#3}}
\crefrangemultiformat{equation}{\textup{#3(#1)#4--#5(#2)#6}}%
{ and \textup{#3(#1)#4--#5(#2)#6}}{, \textup{#3(#1)#4--#5(#2)#6}}{, and \textup{#3(#1)#4--#5(#2)#6}}

% But spell it out at the beginning of a sentence.
\Crefformat{equation}{#2Equation~\textup{(#1)}#3}
\Crefrangeformat{equation}{Equations~\textup{#3(#1)#4--#5(#2)#6}}
\Crefmultiformat{equation}{Equations~\textup{#2(#1)#3}}{ and \textup{#2(#1)#3}}
{, \textup{#2(#1)#3}}{, and \textup{#2(#1)#3}}
\Crefrangemultiformat{equation}{Equations~\textup{#3(#1)#4--#5(#2)#6}}%
{ and \textup{#3(#1)#4--#5(#2)#6}}{, \textup{#3(#1)#4--#5(#2)#6}}{, and \textup{#3(#1)#4--#5(#2)#6}}

% Make number non-italic in any environment.
\crefdefaultlabelformat{#2\textup{#1}#3}

% My name
\author{Jaden Wang}



\begin{document}
\begin{remark}
	Complementary slackness means if $ f_j(x^* )<0$, then this is an inactive constraint, since $ \lambda_j^* =0$, and
	\begin{align*}
		\mathscr{L}(x,\lambda^* ,\nu^* ) &= f_0(x)+ \sum_{i\neq j} \lambda_i^* f_i(x) + \sum \nu_i^* h_i(x) 
	\end{align*}
This is only true if we have strong duality. In particular, it usually isn't true for non-convex problems.
\end{remark}
\begin{eg}[1]
Consider a convex case with inactive constraints:
\begin{align*}
\min_{x \in \rr}\quad &x \\
\text{subject to } \quad &x\geq 0 \\
&x\leq 1 \text{ this is not tight/active constraint} 
\end{align*}
We can just remove the inactive constraint and still get the same solution.
\end{eg}
\begin{eg}[2]
Consider a non-convex case with non-tight constraints:
\begin{align*}
\min_{x \in \rr}\quad &x \\
\text{subject to } \quad &x\geq 0 \qquad \text{ not tight} \\
&x^2 \geq 1
\end{align*}
as solution is $ x^* =1$. But if we remove the non-tight constraint $ x\geq 0$ here, we would get $ -\infty$ as the solution instead, so we can't just drop non-tight constraint for non-convex problems.
\end{eg}

\subsection{Meta-rules}
Suppose $ C \subseteq \rr^{n}$, possibly nonconvex.
\begin{enumerate}[label=(\arabic*)]
	\item switch between min and max with double minus signs or between argmin and argmax with single minus sign (since we don't care about function value).
	\item If $ \phi$ is monotone on $ \im(f)$, then
		\[
			\argmin_{x \in C} \phi(f(x)) = \argmin_{x \in C} f(x)
		.\]
		\begin{eg}
		$ \frac{1}{2} \norm{ Ax-b}^2 $ and $ \norm{ Ax-b} $.
		\end{eg}
	\item If we have all mins or all maxs, we can swap order
		\[
			\min_x \min_y f(x,y) = \min_y \min_x f(x,y) = \min_{x,y} f(x,y)
		.\] 
	\item If $ D \subseteq C$ where $ C$ can be seen as a relaxation, then
		\[
			\min_{x \in C} f(x) \leq \min_{x \in D} f(x)
		.\]
		And we can obtain a lower bound this way.
	\item "superadditivity":
		\[
			\min_{x \in C} f(x)+g(x) \geq \min_{x \in C} f(x) + \min_{x \in C} g(x)
		.\] 
\end{enumerate}

\begin{eg}[solving convex problems using KKT]
	Recall that the solution to the least squares problem $ \min_x \frac{1}{2} \norm{ Ax-b}^2 $ when $ A$ has more rows than columns ($ m\geq n$) is
 \[
	 x^* = \left( A^{T}A \right) ^{-1} A^{T}b
.\]
In the case when $ m<n$, the system is underdetermined so $ Ax=b$ has many solutions, so we instead want to find the solution with the least Euclidean norm (since we can add any vector from $ \ker A$ to arbitrarily inflate the norm of the solution):
\begin{align*}
\min\quad &\frac{1}{2}\norm{ x}^2  \\
\text{subject to } \quad &Ax=b
\end{align*}
And the solution is
\[
	x^*  = A^{T}\left( AA^{T} \right) ^{-1} b
.\] 
To see this, consider the more general quadratic problem
\begin{align*}
	\min\quad & \frac{1}{2} \langle x,Px \rangle + \langle q,x \rangle+ r, P\succeq 0  \\
		  &Ax = b
\end{align*}
Note that we recover the problem when $ P=I, q=r=0$.
The Lagrangian is
\[
	\mathscr{L}(x,\nu) = \frac{1}{2} \langle x,Px \rangle+ \langle q,x \rangle+ r+ \nu^{T}(Ax-b)
.\] 
The KKT conditions are the following:
\begin{enumerate}[label=(\arabic*)]
	\item stationarity:
		\[
			\nabla _x \mathscr{L}(x,\nu) = Px+q+A^{T}\nu = 0
		.\] 
	\item primal feasibility:
		\[
		Ax=b
		.\] 
	\item dual feasibility: N/A.
	\item Complementary slackness: N/A.
\end{enumerate}
Since the conditions are linear equations, we can combine them into a larger system of equations:
\[
	\begin{pmatrix} P&A^{T}\\A&0 \end{pmatrix} \begin{pmatrix} x\\ \nu \end{pmatrix} = \begin{pmatrix} -q\\ b \end{pmatrix}  
.\]
So when $ P=I,q=r=0$, we have
\begin{align*}
	\begin{pmatrix} A&A A^{T} \\ A&0 \end{pmatrix} \begin{pmatrix} x\\ \nu \end{pmatrix}&= \begin{pmatrix} 0\\ b \end{pmatrix}  \\
	\begin{pmatrix} x \\ \nu \end{pmatrix} &= \begin{pmatrix} A^{T}\left(A A^{T} \right) ^{-1} b \\-\left( AA^{T} \right)^{-1} b \end{pmatrix}  
\end{align*}
\end{eg}
\begin{eg}
Consider the problem
\begin{align*}
\min\quad &\frac{1}{2} \norm{ Ax-b}^2  \\
\text{subject to } \quad & \mathbbm{1}^{T} x \leq \tau 
\end{align*}
The Lagrangian is
\[
	\mathscr{L}(x,\lambda) = \frac{1}{2}\norm{ Ax-b}^2 + \lambda (\mathbbm{1}^{T}x - \tau) 
.\]
The KKT conditions are
\begin{enumerate}[label=(\arabic*)]
	\item 
		\[
			\nabla _x \mathscr{L}(x,\lambda) = A^{T}(Ax-b) + \lambda \mathbbm{1} = 0
		.\] 
	\item $ \mathbbm{1}^{T}x - \tau \leq 0$.
	\item $ \lambda\geq 0$.
	\item $ \lambda = 0$ or $ \mathbbm{1}^{T} x -\tau = 0$.
\end{enumerate}
In the case when $ \lambda=0$, the problem reduces to least squares and we've already solved it. When $ \lambda \neq 0$ and $ \mathbbm{1}^{T}x  = r$ instead, we can solve it the following way:
\begin{align*}
	x &= \left( A^{T}A \right) ^{-1} (A^{T}b - \lambda\mathbbm{1}) \\
	\tau=\mathbbm{1}^{T}x &= \mathbbm{1}^{T} \left( A^{T}A \right) ^{-1} (A^{T}b- \lambda \mathbbm{1})\\
	\lambda &= \frac{\mathbbm{1}^{T} \left( A^{T}A \right)^{-1} A^{T}b - \tau }{ \mathbbm{1}^{T} \left( A^{T}A \right)^{-1} \mathbbm{1} } 
\end{align*}
Note that $ \lambda$ is just a scalar.
\end{eg}
\end{document}
