\documentclass[class=article,crop=false]{standalone} 
%Fall 2020
% Some basic packages
\usepackage{standalone}[subpreambles=true]
\usepackage[utf8]{inputenc}
\usepackage[T1]{fontenc}
\usepackage{textcomp}
\usepackage[english]{babel}
\usepackage{url}
\usepackage{graphicx}
\usepackage{float}
\usepackage{enumitem}
\usepackage{lmodern}
\usepackage{hyperref}
\usepackage[usenames,svgnames,dvipsnames]{xcolor}


\pdfminorversion=7

% Don't indent paragraphs, leave some space between them
\usepackage{parskip}

% Hide page number when page is empty
\usepackage{emptypage}
\usepackage{subcaption}
\usepackage{multicol}
\usepackage[dvipsnames]{xcolor}
\usepackage[b]{esvect}

% Math stuff
\usepackage{amsmath, amsfonts, mathtools, amsthm, amssymb}
\usepackage{bbm}

% Fancy script capitals
\usepackage{mathrsfs}
\usepackage{cancel}
% Bold math
\usepackage{bm}
% Some shortcuts
\newcommand{\rr}{\ensuremath{\mathbb{R}}}
\newcommand{\zz}{\ensuremath{\mathbb{Z}}}
\newcommand{\qq}{\ensuremath{\mathbb{Q}}}
\newcommand{\nn}{\ensuremath{\mathbb{N}}}
\newcommand{\ff}{\ensuremath{\mathbb{F}}}
\newcommand{\cc}{\ensuremath{\mathbb{C}}}
\newcommand{\ee}{\ensuremath{\mathbb{E}}}
\renewcommand\O{\ensuremath{\emptyset}}
\newcommand{\norm}[1]{{\left\lVert{#1}\right\rVert}}
\newcommand{\ve}[1]{{\mathbf{#1}}}
\newcommand\allbold[1]{{\boldmath\textbf{#1}}}
\DeclareMathOperator{\lcm}{lcm}
\DeclareMathOperator{\im}{im}
\DeclareMathOperator{\coim}{coim}
\DeclareMathOperator{\dom}{dom}
\DeclareMathOperator{\tr}{tr}
\DeclareMathOperator{\rank}{rank}
\DeclareMathOperator*{\var}{Var}
\DeclareMathOperator*{\ev}{E}
\DeclareMathOperator{\sinc}{sinc}
\DeclareMathOperator{\dg}{deg}
\DeclareMathOperator{\aff}{aff}
\DeclareMathOperator{\conv}{conv}
\DeclareMathOperator{\epi}{epi}
\DeclareMathOperator{\inte}{int}
\DeclareMathOperator{\ri}{ri}
\DeclareMathOperator*{\argmin}{argmin}
\DeclareMathOperator*{\argmax}{argmax}
\DeclareMathOperator{\graph}{graph}
\DeclareMathOperator{\sgn}{sgn}
\DeclareMathOperator*{\Rep}{Rep}
\DeclareMathOperator{\Proj}{Proj}
\DeclareMathOperator{\prox}{prox}
\DeclareMathOperator{\mat}{mat}
\let\vec\relax
\DeclareMathOperator{\vec}{vec}
\let\Re\relax
\DeclareMathOperator{\Re}{Re}
\let\Im\relax
\DeclareMathOperator{\Im}{Im}
% Put x \to \infty below \lim
\let\svlim\lim\def\lim{\svlim\limits}

%wide hat
\usepackage{scalerel,stackengine}
\stackMath
\newcommand*\wh[1]{%
\savestack{\tmpbox}{\stretchto{%
  \scaleto{%
    \scalerel*[\widthof{\ensuremath{#1}}]{\kern-.6pt\bigwedge\kern-.6pt}%
    {\rule[-\textheight/2]{1ex}{\textheight}}%WIDTH-LIMITED BIG WEDGE
  }{\textheight}% 
}{0.5ex}}%
\stackon[1pt]{#1}{\tmpbox}%
}
\parskip 1ex

%Make implies and impliedby shorter
\let\implies\Rightarrow
\let\impliedby\Leftarrow
\let\iff\Leftrightarrow
\let\epsilon\varepsilon

% Add \contra symbol to denote contradiction
\usepackage{stmaryrd} % for \lightning
\newcommand\contra{\scalebox{1.5}{$\lightning$}}

% \let\phi\varphi

% Command for short corrections
% Usage: 1+1=\correct{3}{2}

\definecolor{correct}{HTML}{009900}
\newcommand\correct[2]{\ensuremath{\:}{\color{red}{#1}}\ensuremath{\to }{\color{correct}{#2}}\ensuremath{\:}}
\newcommand\green[1]{{\color{correct}{#1}}}

% horizontal rule
\newcommand\hr{
    \noindent\rule[0.5ex]{\linewidth}{0.5pt}
}

% hide parts
\newcommand\hide[1]{}

% si unitx
\usepackage{siunitx}
\sisetup{locale = FR}

%allows pmatrix to stretch
\makeatletter
\renewcommand*\env@matrix[1][\arraystretch]{%
  \edef\arraystretch{#1}%
  \hskip -\arraycolsep
  \let\@ifnextchar\new@ifnextchar
  \array{*\c@MaxMatrixCols c}}
\makeatother

\renewcommand{\arraystretch}{0.8}

% Environments
\makeatother
% For box around Definition, Theorem, \ldots
%%fakesection Theorems
\usepackage{thmtools}
\usepackage[framemethod=TikZ]{mdframed}

\theoremstyle{definition}
\mdfdefinestyle{mdbluebox}{%
	roundcorner = 10pt,
	linewidth=1pt,
	skipabove=12pt,
	innerbottommargin=9pt,
	skipbelow=2pt,
	nobreak=true,
	linecolor=blue,
	backgroundcolor=TealBlue!5,
}
\declaretheoremstyle[
	headfont=\sffamily\bfseries\color{MidnightBlue},
	mdframed={style=mdbluebox},
	headpunct={\\[3pt]},
	postheadspace={0pt}
]{thmbluebox}

\mdfdefinestyle{mdredbox}{%
	linewidth=0.5pt,
	skipabove=12pt,
	frametitleaboveskip=5pt,
	frametitlebelowskip=0pt,
	skipbelow=2pt,
	frametitlefont=\bfseries,
	innertopmargin=4pt,
	innerbottommargin=8pt,
	nobreak=false,
	linecolor=RawSienna,
	backgroundcolor=Salmon!5,
}
\declaretheoremstyle[
	headfont=\bfseries\color{RawSienna},
	mdframed={style=mdredbox},
	headpunct={\\[3pt]},
	postheadspace={0pt},
]{thmredbox}

\declaretheorem[%
style=thmbluebox,name=Theorem,numberwithin=section]{thm}
\declaretheorem[style=thmbluebox,name=Lemma,sibling=thm]{lem}
\declaretheorem[style=thmbluebox,name=Proposition,sibling=thm]{prop}
\declaretheorem[style=thmbluebox,name=Corollary,sibling=thm]{coro}
\declaretheorem[style=thmredbox,name=Example,sibling=thm]{eg}

\mdfdefinestyle{mdgreenbox}{%
	roundcorner = 10pt,
	linewidth=1pt,
	skipabove=12pt,
	innerbottommargin=9pt,
	skipbelow=2pt,
	nobreak=true,
	linecolor=ForestGreen,
	backgroundcolor=ForestGreen!5,
}

\declaretheoremstyle[
	headfont=\bfseries\sffamily\color{ForestGreen!70!black},
	bodyfont=\normalfont,
	spaceabove=2pt,
	spacebelow=1pt,
	mdframed={style=mdgreenbox},
	headpunct={ --- },
]{thmgreenbox}

\declaretheorem[style=thmgreenbox,name=Definition,sibling=thm]{defn}

\mdfdefinestyle{mdgreenboxsq}{%
	linewidth=1pt,
	skipabove=12pt,
	innerbottommargin=9pt,
	skipbelow=2pt,
	nobreak=true,
	linecolor=ForestGreen,
	backgroundcolor=ForestGreen!5,
}
\declaretheoremstyle[
	headfont=\bfseries\sffamily\color{ForestGreen!70!black},
	bodyfont=\normalfont,
	spaceabove=2pt,
	spacebelow=1pt,
	mdframed={style=mdgreenboxsq},
	headpunct={},
]{thmgreenboxsq}
\declaretheoremstyle[
	headfont=\bfseries\sffamily\color{ForestGreen!70!black},
	bodyfont=\normalfont,
	spaceabove=2pt,
	spacebelow=1pt,
	mdframed={style=mdgreenboxsq},
	headpunct={},
]{thmgreenboxsq*}

\mdfdefinestyle{mdblackbox}{%
	skipabove=8pt,
	linewidth=3pt,
	rightline=false,
	leftline=true,
	topline=false,
	bottomline=false,
	linecolor=black,
	backgroundcolor=RedViolet!5!gray!5,
}
\declaretheoremstyle[
	headfont=\bfseries,
	bodyfont=\normalfont\small,
	spaceabove=0pt,
	spacebelow=0pt,
	mdframed={style=mdblackbox}
]{thmblackbox}

\theoremstyle{plain}
\declaretheorem[name=Question,sibling=thm,style=thmblackbox]{ques}
\declaretheorem[name=Remark,sibling=thm,style=thmgreenboxsq]{remark}
\declaretheorem[name=Remark,sibling=thm,style=thmgreenboxsq*]{remark*}

\theoremstyle{definition}
\newtheorem{claim}[thm]{Claim}
\theoremstyle{remark}
\newtheorem*{case}{Case}
\newtheorem*{notation}{Notation}
\newtheorem*{note}{Note}
\newtheorem*{motivation}{Motivation}
\newtheorem*{intuition}{Intuition}

% Make section starts with 1 for report type
%\renewcommand\thesection{\arabic{section}}

% End example and intermezzo environments with a small diamond (just like proof
% environments end with a small square)
\usepackage{etoolbox}
\AtEndEnvironment{vb}{\null\hfill$\diamond$}%
\AtEndEnvironment{intermezzo}{\null\hfill$\diamond$}%
% \AtEndEnvironment{opmerking}{\null\hfill$\diamond$}%

% Fix some spacing
% http://tex.stackexchange.com/questions/22119/how-can-i-change-the-spacing-before-theorems-with-amsthm
\makeatletter
\def\thm@space@setup{%
  \thm@preskip=\parskip \thm@postskip=0pt
}

% Fix some stuff
% %http://tex.stackexchange.com/questions/76273/multiple-pdfs-with-page-group-included-in-a-single-page-warning
\pdfsuppresswarningpagegroup=1

\renewcommand{\baselinestretch}{1.5}
\RequirePackage{hyperref}[6.83]
\hypersetup{
  colorlinks=false,
  frenchlinks=false,
  pdfborder={0 0 0},
  naturalnames=false,
  hypertexnames=false,
  breaklinks
}
\urlstyle{same}

\usepackage{graphics}
\usepackage{epstopdf}

%%
%% Add support for color in order to color the hyperlinks.
%% 
\hypersetup{
  colorlinks = true,
  allcolors = siaminlinkcolor,
  urlcolor = siamexlinkcolor,
}
%%fakesection Links
\hypersetup{
    colorlinks,
    linkcolor={red!50!black},
    citecolor={green!50!black},
    urlcolor={blue!80!black}
}
%customization of cleveref
\RequirePackage[capitalize,nameinlink]{cleveref}[0.19]

% Per SIAM Style Manual, "section" should be lowercase
\crefname{section}{section}{sections}
\crefname{subsection}{subsection}{subsections}
\Crefname{section}{Section}{Sections}
\Crefname{subsection}{Subsection}{Subsections}

% Per SIAM Style Manual, "Figure" should be spelled out in references
\Crefname{figure}{Figure}{Figures}

% Per SIAM Style Manual, don't say equation in front on an equation.
\crefformat{equation}{\textup{#2(#1)#3}}
\crefrangeformat{equation}{\textup{#3(#1)#4--#5(#2)#6}}
\crefmultiformat{equation}{\textup{#2(#1)#3}}{ and \textup{#2(#1)#3}}
{, \textup{#2(#1)#3}}{, and \textup{#2(#1)#3}}
\crefrangemultiformat{equation}{\textup{#3(#1)#4--#5(#2)#6}}%
{ and \textup{#3(#1)#4--#5(#2)#6}}{, \textup{#3(#1)#4--#5(#2)#6}}{, and \textup{#3(#1)#4--#5(#2)#6}}

% But spell it out at the beginning of a sentence.
\Crefformat{equation}{#2Equation~\textup{(#1)}#3}
\Crefrangeformat{equation}{Equations~\textup{#3(#1)#4--#5(#2)#6}}
\Crefmultiformat{equation}{Equations~\textup{#2(#1)#3}}{ and \textup{#2(#1)#3}}
{, \textup{#2(#1)#3}}{, and \textup{#2(#1)#3}}
\Crefrangemultiformat{equation}{Equations~\textup{#3(#1)#4--#5(#2)#6}}%
{ and \textup{#3(#1)#4--#5(#2)#6}}{, \textup{#3(#1)#4--#5(#2)#6}}{, and \textup{#3(#1)#4--#5(#2)#6}}

% Make number non-italic in any environment.
\crefdefaultlabelformat{#2\textup{#1}#3}

% My name
\author{Jaden Wang}



\begin{document}
\subsubsection{Recovering a primal solution from a dual solution}
\begin{thm}[BC17 19.1]
	Let $ f \in \Gamma_0(\rr^{n}), g \in \Gamma_0(\rr^{m}), \dom(g) \cap A(\dom f) \neq \O$. The following are equivalent:
	\begin{enumerate}[label=(\arabic*)]
		\item There is no duality gap, and $ x,v$ are primal-dual optimal.  \emph{i.e.} there exists saddle points to $ \mathscr{L}(x,v)=f(x) + \langle Ax,v \rangle - g^* (v)$.
		\item $ A^* v \in \partial f(x)$ and $ -v \in \partial g(Ax)$.
		\item $ x \in \partial f^* (A^* v)$ and $ Ax \in \partial g^* (-v)$.
	\end{enumerate}
\end{thm}
Note that $ 2 \iff 3$ since $ \partial f^* = \partial f^{-1}$ when $ f \in \Gamma_0$.
\begin{prop}[BC17 19.4]
	Under above conditions, if $ f^* $ is differentiable at $ (A^* v)$ (\emph{i.e.} if $ f$ is strictly convex), then either
	\begin{enumerate}[label=(\alph*)]
		\item there is no primal optimal solution, or
		\item $ x = \nabla f^* (A^* v)$ is primal optimal.
	\end{enumerate}
\end{prop}

Let's see a simple example applying Fenchel-Rockafellar duality:
\begin{eg}
\begin{align*}
	(P) \qquad \qquad \min\quad & \frac{1}{2} \norm{ x-x_0}^2  \\
\text{subject to } \quad &\norm{ Ax-b}\leq \epsilon  
\end{align*}
This is hard to solve directly using gradient descent, as we would need to project which requires finding the SVD of A.
Let $ f(x) = \frac{1}{2} \norm{ x-x_0}^2 $ and $ g(y) = I_{\norm{ y-b}\leq \epsilon }$. Then
\begin{align*}
	f^* (v) &= \sup_x \langle v,x \rangle-\frac{1}{2}\norm{ x-x_0} ^2\\
	&= \langle v,x_0 \rangle+ \sup_{\widetilde{ x}} \langle v,\widetilde{ x} \rangle -\frac{1}{2} \norm{ \widetilde{ x}}^2  \\
	&= \langle v,x_0 \rangle + \frac{1}{2} \norm{ v}^2  \\
	g^* (v) &= \sup_{\norm{ y-b}\leq \epsilon } \langle v,y \rangle \\
	&= \langle v,b \rangle+ \sup_{ \norm{ \widetilde{ y}}\leq \epsilon } \langle v,\widetilde{ y} \rangle \\
	&= \langle v,b \rangle + \epsilon \norm{ v}_2  
\end{align*}
Hence we obtain the dual:
\[
	(D) \qquad \min_v \left( \underbrace{ \langle A^* v,x_0 \rangle + \frac{1}{2} \norm{ A^* v}_2^2 }_{f^* (A^* v) } \right) + \left( \underbrace{ \langle v,-b \rangle +  \epsilon \norm{ v}_2 }_{ g^* (-v)} \right)   
,\]
where the first three terms should be differentiable and the last term should have easy-to-find proximity operator.
\end{eg}

\subsection{Optimality conditions}
\subsubsection{Complementary slackness}
Suppose we have primal and dual optimal solutions, $ x^* ,\lambda^* ,\nu^* $, and have strong duality (no need for convexity). That is, we assume saddle points exist. Observe
\begin{align*}
	p^* =\qquad \min\quad &f_0(x) = f_0(x^* ) = g(\lambda^* ,\nu^* ) = \inf_x \mathscr{L}(x,\lambda^* ,\nu^* )\\
\text{subject to } \quad &f_i(x) \leq 0, i = 1,\ldots,m \\
&h_i(x) = 0 , i = 1,\ldots,p
\end{align*}
Thus,
\begin{align*}
	p^* &\leq \mathscr{L}(x^* ,\lambda^* ,\nu^* )\\
	    &= f_0(x^* )+ \sum \underbrace{ \lambda_i^*}_{\geq 0} \underbrace{f_i(x^* )}_{\leq 0} + \underbrace{ \sum \nu_i^* h_i(x^* )}_{=0} \\
	    &\leq f_0 (x^* ) = p^*   
\end{align*}
Hence, for all $ i$,  $ \lambda_i^* f_i(x^* ) = 0$. This is called \allbold{complementary slackness}. That is, either $ \lambda_i^* =0$ or $ f_i(x^* ) = 0$.

If $ f_i(x^* )<0$, then $ \lambda_i^* $ must be zero, meaning that the constraints have no effect on the solution (not tight).

\subsubsection{KKT conditions}
~\begin{thm}[KKT 1 "necessary"]
	Suppose $ f_i,h_i$ are differentiable. If $ x^* $ is primal optimal, and $ \lambda^*,\nu^*  $ dual optimal, and no duality gap (strong duality), then necessarily $ x^* ,\lambda^* ,\nu^* $ satisfy
	\begin{enumerate}[label=(\arabic*)]
		\item stationarity: 
			 \[
				 0 = \nabla f_0(x^* ) + \sum \lambda_i^* f_i(x^* ) + \sum \nu_i h_i(x^* ) = \nabla _x \mathscr{L}(x^* ,\lambda^* ,\nu^* )
			.\] 
		\item primal feasibility: $ f_i(x^* )\leq 0, h_i(x^* ) =0$.
		\item dual feasibility: $ \lambda^* \geq 0$.
		\item complementary slackness: $ \lambda_i^* f_i(x^* ) =0$.
	\end{enumerate}
\end{thm}
\begin{note}
There is no need for convexity here.
\end{note}
\begin{remark}
~\begin{enumerate}[label=(\alph*)]
	\item These are only necessary for saddle points and may not be enough without saddle points or strong duality.
		\begin{eg}
		$ \min_x e^{-x}$ satisfies the conditions but has no saddle points. It in fact doesn't have a minimizer. 
		\end{eg}
	\item only needed if differentiable.
	\item If functions are nonconvex, these may not be sufficient. That is, there may be non-optimal solutions.
\end{enumerate}
\end{remark}
\begin{remark}
If the primal problem is convex, then the existence of saddle points guarantees strong duality.
\end{remark}

We can generalize stationarity:
\begin{enumerate}[label=(\arabic*)]
	\item $ x^*  \in \argmin_x \mathscr{L}(x,\lambda^* ,\nu^* )$.

			By convexity, we have
			\[
				0 \in \partial \mathscr{L}(x^* ,\lambda^* ,\nu^* )
			.\] 
			If we have CQ,
			\[
				0 \in \partial f_0(x^* ) + \sum \lambda_i^* \partial f_i(x^* ) + \sum \partial h_i(x^* )
			.\] 
\end{enumerate}

\begin{thm}[KKT 2 sufficient]
	Suppose $ (P)$ is convex ($ f_i$ are convex, $ h_i$ are affine). If $ (x^* ,\lambda^* ,\nu^* )$ solve the KKT conditions, then they are the primal and dual optimal and there is no duality gap.
\end{thm}
\begin{proof}
	Assume $ (x^* ,\lambda^* ,\nu^* )$ satisfies the KKT conditions.
	\begin{align*}
		p^* &\leq f_0(x^* ) \qquad  x^* \text{is feasible }\\
		    &= \mathscr{L}(x^* ,\lambda^* ,\nu^* ) \qquad \text{ since feasible and comp. slack.} \\
		    &\inf_x \mathscr{L}(x,\lambda^* ,\nu^* ) \qquad \text{ stationarity}  \\
		    &= g(alm^* ,\nu^* ) \qquad  \text{ dual feasibility} \\
		    &\leq d^*  
	\end{align*}
	Thus we prove strong duality, and $ p^* =f_0(x^* ) \implies x^* $ is optimal. Same for duals. 
\end{proof}
\end{document}
