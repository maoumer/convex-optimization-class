\documentclass[class=article,crop=false]{standalone} 
%Fall 2020
% Some basic packages
\usepackage{standalone}[subpreambles=true]
\usepackage[utf8]{inputenc}
\usepackage[T1]{fontenc}
\usepackage{textcomp}
\usepackage[english]{babel}
\usepackage{url}
\usepackage{graphicx}
\usepackage{float}
\usepackage{enumitem}
\usepackage{lmodern}
\usepackage{hyperref}
\usepackage[usenames,svgnames,dvipsnames]{xcolor}


\pdfminorversion=7

% Don't indent paragraphs, leave some space between them
\usepackage{parskip}

% Hide page number when page is empty
\usepackage{emptypage}
\usepackage{subcaption}
\usepackage{multicol}
\usepackage[dvipsnames]{xcolor}
\usepackage[b]{esvect}

% Math stuff
\usepackage{amsmath, amsfonts, mathtools, amsthm, amssymb}
\usepackage{bbm}

% Fancy script capitals
\usepackage{mathrsfs}
\usepackage{cancel}
% Bold math
\usepackage{bm}
% Some shortcuts
\newcommand{\rr}{\ensuremath{\mathbb{R}}}
\newcommand{\zz}{\ensuremath{\mathbb{Z}}}
\newcommand{\qq}{\ensuremath{\mathbb{Q}}}
\newcommand{\nn}{\ensuremath{\mathbb{N}}}
\newcommand{\ff}{\ensuremath{\mathbb{F}}}
\newcommand{\cc}{\ensuremath{\mathbb{C}}}
\newcommand{\ee}{\ensuremath{\mathbb{E}}}
\renewcommand\O{\ensuremath{\emptyset}}
\newcommand{\norm}[1]{{\left\lVert{#1}\right\rVert}}
\newcommand{\ve}[1]{{\mathbf{#1}}}
\newcommand\allbold[1]{{\boldmath\textbf{#1}}}
\DeclareMathOperator{\lcm}{lcm}
\DeclareMathOperator{\im}{im}
\DeclareMathOperator{\coim}{coim}
\DeclareMathOperator{\dom}{dom}
\DeclareMathOperator{\tr}{tr}
\DeclareMathOperator{\rank}{rank}
\DeclareMathOperator*{\var}{Var}
\DeclareMathOperator*{\ev}{E}
\DeclareMathOperator{\sinc}{sinc}
\DeclareMathOperator{\dg}{deg}
\DeclareMathOperator{\aff}{aff}
\DeclareMathOperator{\conv}{conv}
\DeclareMathOperator{\epi}{epi}
\DeclareMathOperator{\inte}{int}
\DeclareMathOperator{\ri}{ri}
\DeclareMathOperator*{\argmin}{argmin}
\DeclareMathOperator*{\argmax}{argmax}
\DeclareMathOperator{\graph}{graph}
\DeclareMathOperator{\sgn}{sgn}
\DeclareMathOperator*{\Rep}{Rep}
\DeclareMathOperator{\Proj}{Proj}
\DeclareMathOperator{\prox}{prox}
\DeclareMathOperator{\mat}{mat}
\let\vec\relax
\DeclareMathOperator{\vec}{vec}
\let\Re\relax
\DeclareMathOperator{\Re}{Re}
\let\Im\relax
\DeclareMathOperator{\Im}{Im}
% Put x \to \infty below \lim
\let\svlim\lim\def\lim{\svlim\limits}

%wide hat
\usepackage{scalerel,stackengine}
\stackMath
\newcommand*\wh[1]{%
\savestack{\tmpbox}{\stretchto{%
  \scaleto{%
    \scalerel*[\widthof{\ensuremath{#1}}]{\kern-.6pt\bigwedge\kern-.6pt}%
    {\rule[-\textheight/2]{1ex}{\textheight}}%WIDTH-LIMITED BIG WEDGE
  }{\textheight}% 
}{0.5ex}}%
\stackon[1pt]{#1}{\tmpbox}%
}
\parskip 1ex

%Make implies and impliedby shorter
\let\implies\Rightarrow
\let\impliedby\Leftarrow
\let\iff\Leftrightarrow
\let\epsilon\varepsilon

% Add \contra symbol to denote contradiction
\usepackage{stmaryrd} % for \lightning
\newcommand\contra{\scalebox{1.5}{$\lightning$}}

% \let\phi\varphi

% Command for short corrections
% Usage: 1+1=\correct{3}{2}

\definecolor{correct}{HTML}{009900}
\newcommand\correct[2]{\ensuremath{\:}{\color{red}{#1}}\ensuremath{\to }{\color{correct}{#2}}\ensuremath{\:}}
\newcommand\green[1]{{\color{correct}{#1}}}

% horizontal rule
\newcommand\hr{
    \noindent\rule[0.5ex]{\linewidth}{0.5pt}
}

% hide parts
\newcommand\hide[1]{}

% si unitx
\usepackage{siunitx}
\sisetup{locale = FR}

%allows pmatrix to stretch
\makeatletter
\renewcommand*\env@matrix[1][\arraystretch]{%
  \edef\arraystretch{#1}%
  \hskip -\arraycolsep
  \let\@ifnextchar\new@ifnextchar
  \array{*\c@MaxMatrixCols c}}
\makeatother

\renewcommand{\arraystretch}{0.8}

% Environments
\makeatother
% For box around Definition, Theorem, \ldots
%%fakesection Theorems
\usepackage{thmtools}
\usepackage[framemethod=TikZ]{mdframed}

\theoremstyle{definition}
\mdfdefinestyle{mdbluebox}{%
	roundcorner = 10pt,
	linewidth=1pt,
	skipabove=12pt,
	innerbottommargin=9pt,
	skipbelow=2pt,
	nobreak=true,
	linecolor=blue,
	backgroundcolor=TealBlue!5,
}
\declaretheoremstyle[
	headfont=\sffamily\bfseries\color{MidnightBlue},
	mdframed={style=mdbluebox},
	headpunct={\\[3pt]},
	postheadspace={0pt}
]{thmbluebox}

\mdfdefinestyle{mdredbox}{%
	linewidth=0.5pt,
	skipabove=12pt,
	frametitleaboveskip=5pt,
	frametitlebelowskip=0pt,
	skipbelow=2pt,
	frametitlefont=\bfseries,
	innertopmargin=4pt,
	innerbottommargin=8pt,
	nobreak=false,
	linecolor=RawSienna,
	backgroundcolor=Salmon!5,
}
\declaretheoremstyle[
	headfont=\bfseries\color{RawSienna},
	mdframed={style=mdredbox},
	headpunct={\\[3pt]},
	postheadspace={0pt},
]{thmredbox}

\declaretheorem[%
style=thmbluebox,name=Theorem,numberwithin=section]{thm}
\declaretheorem[style=thmbluebox,name=Lemma,sibling=thm]{lem}
\declaretheorem[style=thmbluebox,name=Proposition,sibling=thm]{prop}
\declaretheorem[style=thmbluebox,name=Corollary,sibling=thm]{coro}
\declaretheorem[style=thmredbox,name=Example,sibling=thm]{eg}

\mdfdefinestyle{mdgreenbox}{%
	roundcorner = 10pt,
	linewidth=1pt,
	skipabove=12pt,
	innerbottommargin=9pt,
	skipbelow=2pt,
	nobreak=true,
	linecolor=ForestGreen,
	backgroundcolor=ForestGreen!5,
}

\declaretheoremstyle[
	headfont=\bfseries\sffamily\color{ForestGreen!70!black},
	bodyfont=\normalfont,
	spaceabove=2pt,
	spacebelow=1pt,
	mdframed={style=mdgreenbox},
	headpunct={ --- },
]{thmgreenbox}

\declaretheorem[style=thmgreenbox,name=Definition,sibling=thm]{defn}

\mdfdefinestyle{mdgreenboxsq}{%
	linewidth=1pt,
	skipabove=12pt,
	innerbottommargin=9pt,
	skipbelow=2pt,
	nobreak=true,
	linecolor=ForestGreen,
	backgroundcolor=ForestGreen!5,
}
\declaretheoremstyle[
	headfont=\bfseries\sffamily\color{ForestGreen!70!black},
	bodyfont=\normalfont,
	spaceabove=2pt,
	spacebelow=1pt,
	mdframed={style=mdgreenboxsq},
	headpunct={},
]{thmgreenboxsq}
\declaretheoremstyle[
	headfont=\bfseries\sffamily\color{ForestGreen!70!black},
	bodyfont=\normalfont,
	spaceabove=2pt,
	spacebelow=1pt,
	mdframed={style=mdgreenboxsq},
	headpunct={},
]{thmgreenboxsq*}

\mdfdefinestyle{mdblackbox}{%
	skipabove=8pt,
	linewidth=3pt,
	rightline=false,
	leftline=true,
	topline=false,
	bottomline=false,
	linecolor=black,
	backgroundcolor=RedViolet!5!gray!5,
}
\declaretheoremstyle[
	headfont=\bfseries,
	bodyfont=\normalfont\small,
	spaceabove=0pt,
	spacebelow=0pt,
	mdframed={style=mdblackbox}
]{thmblackbox}

\theoremstyle{plain}
\declaretheorem[name=Question,sibling=thm,style=thmblackbox]{ques}
\declaretheorem[name=Remark,sibling=thm,style=thmgreenboxsq]{remark}
\declaretheorem[name=Remark,sibling=thm,style=thmgreenboxsq*]{remark*}

\theoremstyle{definition}
\newtheorem{claim}[thm]{Claim}
\theoremstyle{remark}
\newtheorem*{case}{Case}
\newtheorem*{notation}{Notation}
\newtheorem*{note}{Note}
\newtheorem*{motivation}{Motivation}
\newtheorem*{intuition}{Intuition}

% Make section starts with 1 for report type
%\renewcommand\thesection{\arabic{section}}

% End example and intermezzo environments with a small diamond (just like proof
% environments end with a small square)
\usepackage{etoolbox}
\AtEndEnvironment{vb}{\null\hfill$\diamond$}%
\AtEndEnvironment{intermezzo}{\null\hfill$\diamond$}%
% \AtEndEnvironment{opmerking}{\null\hfill$\diamond$}%

% Fix some spacing
% http://tex.stackexchange.com/questions/22119/how-can-i-change-the-spacing-before-theorems-with-amsthm
\makeatletter
\def\thm@space@setup{%
  \thm@preskip=\parskip \thm@postskip=0pt
}

% Fix some stuff
% %http://tex.stackexchange.com/questions/76273/multiple-pdfs-with-page-group-included-in-a-single-page-warning
\pdfsuppresswarningpagegroup=1

\renewcommand{\baselinestretch}{1.5}
\RequirePackage{hyperref}[6.83]
\hypersetup{
  colorlinks=false,
  frenchlinks=false,
  pdfborder={0 0 0},
  naturalnames=false,
  hypertexnames=false,
  breaklinks
}
\urlstyle{same}

\usepackage{graphics}
\usepackage{epstopdf}

%%
%% Add support for color in order to color the hyperlinks.
%% 
\hypersetup{
  colorlinks = true,
  allcolors = siaminlinkcolor,
  urlcolor = siamexlinkcolor,
}
%%fakesection Links
\hypersetup{
    colorlinks,
    linkcolor={red!50!black},
    citecolor={green!50!black},
    urlcolor={blue!80!black}
}
%customization of cleveref
\RequirePackage[capitalize,nameinlink]{cleveref}[0.19]

% Per SIAM Style Manual, "section" should be lowercase
\crefname{section}{section}{sections}
\crefname{subsection}{subsection}{subsections}
\Crefname{section}{Section}{Sections}
\Crefname{subsection}{Subsection}{Subsections}

% Per SIAM Style Manual, "Figure" should be spelled out in references
\Crefname{figure}{Figure}{Figures}

% Per SIAM Style Manual, don't say equation in front on an equation.
\crefformat{equation}{\textup{#2(#1)#3}}
\crefrangeformat{equation}{\textup{#3(#1)#4--#5(#2)#6}}
\crefmultiformat{equation}{\textup{#2(#1)#3}}{ and \textup{#2(#1)#3}}
{, \textup{#2(#1)#3}}{, and \textup{#2(#1)#3}}
\crefrangemultiformat{equation}{\textup{#3(#1)#4--#5(#2)#6}}%
{ and \textup{#3(#1)#4--#5(#2)#6}}{, \textup{#3(#1)#4--#5(#2)#6}}{, and \textup{#3(#1)#4--#5(#2)#6}}

% But spell it out at the beginning of a sentence.
\Crefformat{equation}{#2Equation~\textup{(#1)}#3}
\Crefrangeformat{equation}{Equations~\textup{#3(#1)#4--#5(#2)#6}}
\Crefmultiformat{equation}{Equations~\textup{#2(#1)#3}}{ and \textup{#2(#1)#3}}
{, \textup{#2(#1)#3}}{, and \textup{#2(#1)#3}}
\Crefrangemultiformat{equation}{Equations~\textup{#3(#1)#4--#5(#2)#6}}%
{ and \textup{#3(#1)#4--#5(#2)#6}}{, \textup{#3(#1)#4--#5(#2)#6}}{, and \textup{#3(#1)#4--#5(#2)#6}}

% Make number non-italic in any environment.
\crefdefaultlabelformat{#2\textup{#1}#3}

% My name
\author{Jaden Wang}



\begin{document}
\newpage
\subsection{Linear conjugate gradient method}
CG solves (usually approximately) $ Ax =b$ if  $ A \succ 0$. More details and intuition can be found at \url{cs.cmu.edu/~quake-papers/painless-conjugate-gradient.pdf}.

\begin{eg}
	Consider the least squares problem. Let $ \phi(x) = \frac{1}{2} x^{T} \widetilde{ A}^{T} \widetilde{ A}x - \widetilde{ b}^{T} \widetilde{ A}x + \frac{1}{2} \widetilde{ b}^{T}\widetilde{ b} =: \frac{1}{2} x^{T} A x - b^{T} x + \text{ const.} $. Here $ A \succeq 0$ always and $ A \succ 0$ if $ m>n$ and full rank. Then we can solve  $ \nabla \phi(x) = Ax-b$.
	\begin{note}
	Don't form the Gram matrix. Instead use LSQR method.
	\end{note}

	One idea to $ \min \phi(x)$ is coordinate descent/alternating minimization. This slowly converges to the solution via a zigzag path. If we change to the eigenvector basis, it is guaranteed to converge in $ n$ steps. However, finding the eigenvector basis is just as expensive as solving the normal equation directly at  $ \mathcal{ O}(n^3)$.
\begin{defn}[conjugate directions]
$ \{p_i\} $ are conjugate directions if they are $ A$-orthogonal. That is, 
\[
\langle p_i |A|p_j \rangle := \langle p_i, A p_j \rangle = 0 \text{ if } i\neq j 
.\] 
\end{defn}

\begin{note}
If we have $ \{p_i\}_{i=0}^{n-1} $, it's a basis. If $ p_i$ are eigenvectors of a symmetric matrix  $ A$, then they are  $ A$-orthogonal.
\end{note}

Our goal is to find $ \{p_i\} $ more cheaply than eigenvectors.
\end{eg}

\begin{thm}[conjugate direction method (abstract)]
Assume $ \{p_i\}_{i=0}^{n-1} $ are conjugate directions. Then
\begin{align*}
	x_{k+1} = x_k + \alpha_k p_k 
\end{align*}
where $ a_k$ solves $ \min_{\alpha} \phi(x_k + \alpha p_k)$ which is exact line search. The solution to this 1D problem has a closed form:
\begin{align*}
	a_k = - \frac{\langle r_k | p_k \rangle}{ \langle p_k|A|p_k \rangle}
\end{align*}
where $ r_k = A x_k-b$. Then $ x_n = x^* $.
\end{thm}

\begin{proof}
Since $ \{ p_i\} $ is a basis, we can write
\begin{align*}
	x^* -x_0 &= \sum_{ i= 0}^{ n-1} \sigma_i p_i \\
	p_k^{T} A(x^* -x_0)&= \sum_{ i= 0}^{ n-1} \sigma_i \langle p_k|A|p_i \rangle \\
	&= \sigma_k \langle p_k|A|p_k \rangle \\
	\sigma_k &= \frac{\langle p_k|A|x^* -x_0 \rangle}{ \langle p_k|A|p_k \rangle} \ \forall \ k
\end{align*}
Moreover,
\begin{align*}
	x_k - x_0 &= \sum_{ i= 0}^{ k-1} \alpha_i p_i \\
	p_k^{T} A(x_k - x_0) &= \sum_{ i= 0}^{ k-1} \alpha_i \langle p_k|A|p_i \rangle \\
	\langle p_k|A|x_k-x_0 \rangle &= 0
\end{align*}
Substituting $ x_k $ as $ x_0$,
\begin{align*}
	\sigma_k = \frac{\langle p_k|A|x^* -x_k \rangle}{ \langle p_k|A|p_k \rangle} = \alpha_k
\end{align*}
Therefore, $ x_n = x^* $ since they have the same expression in the basis.
\end{proof}
\begin{remark}
We can think of this process as either building up $ x^* $ component-by-component or cutting the error  $ x^* -x_k$ component-by-component.
\end{remark}
Facts:
\begin{itemize}
	\item $ r_{k+1} = r_k + \alpha_k A p_k$ 
	\item $ \langle r_k, p_i \rangle=0, i<k$.
	\item $ x_k$ minimizes $ \phi$ over $ K(r_0,k) = \text{ span}\{r_0, A r_0,\ldots,A^{k-1} r_0\}  $. This is the Krylov subspace. Note that $ K(r_0,n-1) = \rr^{n}$.
	\item $ \langle r_k,r_i \rangle=0, i<k$.
	\item $ p_k, r_k \in K(r_0,k)$.
\end{itemize}

\begin{thm}[conjugate gradient]
Given arbitrary $ x_0$, $ r_0 = A x_0 - b$, $ p_0 = -r_0$. Compute iteratively
\begin{align*}
	\beta_k &= \frac{\langle r_k|A|p_{k-1} \rangle}{ \langle p_{k-1}|A|p_{k-1} \rangle}, \text{ chosen s.t. }  \langle p_k|A|p_{k-1} \rangle=0 \\
p_k &= -r_k + \beta_k p_{k-1}\\
\alpha_k &= - \frac{\langle r_k|p_k \rangle}{ \langle p_k|A|p_k \rangle} \\
	x_{k+1} &= x_k + \alpha_k p_k \\
	r_{k+1} &= r_k - \alpha_k A p_k
\end{align*}
The magic is that $ \langle p_k|A|p_i \rangle=0 \ \forall \ i\leq k-1$ (see Nocedal and Wright for proof). 

The cost is one matrix-vector multiply per step.
\end{thm}
\begin{thm}[convergence of CG]
\begin{align*}
	\norm{ x_k - x^* }_A \leq 2 \left( \frac{\sqrt{\kappa}-1 }{\sqrt{\kappa} +1 } \right)^{k} \norm{ x_0 - x^* }_A 
\end{align*}
where $ \kappa = \kappa(A)$ is the condition number of  $ A$.
\end{thm}
\end{document}
