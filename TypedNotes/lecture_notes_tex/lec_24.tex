\documentclass[class=article,crop=false]{standalone} 
%Fall 2020
% Some basic packages
\usepackage{standalone}[subpreambles=true]
\usepackage[utf8]{inputenc}
\usepackage[T1]{fontenc}
\usepackage{textcomp}
\usepackage[english]{babel}
\usepackage{url}
\usepackage{graphicx}
\usepackage{float}
\usepackage{enumitem}
\usepackage{lmodern}
\usepackage{hyperref}
\usepackage[usenames,svgnames,dvipsnames]{xcolor}


\pdfminorversion=7

% Don't indent paragraphs, leave some space between them
\usepackage{parskip}

% Hide page number when page is empty
\usepackage{emptypage}
\usepackage{subcaption}
\usepackage{multicol}
\usepackage[dvipsnames]{xcolor}
\usepackage[b]{esvect}

% Math stuff
\usepackage{amsmath, amsfonts, mathtools, amsthm, amssymb}
\usepackage{bbm}

% Fancy script capitals
\usepackage{mathrsfs}
\usepackage{cancel}
% Bold math
\usepackage{bm}
% Some shortcuts
\newcommand{\rr}{\ensuremath{\mathbb{R}}}
\newcommand{\zz}{\ensuremath{\mathbb{Z}}}
\newcommand{\qq}{\ensuremath{\mathbb{Q}}}
\newcommand{\nn}{\ensuremath{\mathbb{N}}}
\newcommand{\ff}{\ensuremath{\mathbb{F}}}
\newcommand{\cc}{\ensuremath{\mathbb{C}}}
\newcommand{\ee}{\ensuremath{\mathbb{E}}}
\renewcommand\O{\ensuremath{\emptyset}}
\newcommand{\norm}[1]{{\left\lVert{#1}\right\rVert}}
\newcommand{\ve}[1]{{\mathbf{#1}}}
\newcommand\allbold[1]{{\boldmath\textbf{#1}}}
\DeclareMathOperator{\lcm}{lcm}
\DeclareMathOperator{\im}{im}
\DeclareMathOperator{\coim}{coim}
\DeclareMathOperator{\dom}{dom}
\DeclareMathOperator{\tr}{tr}
\DeclareMathOperator{\rank}{rank}
\DeclareMathOperator*{\var}{Var}
\DeclareMathOperator*{\ev}{E}
\DeclareMathOperator{\sinc}{sinc}
\DeclareMathOperator{\dg}{deg}
\DeclareMathOperator{\aff}{aff}
\DeclareMathOperator{\conv}{conv}
\DeclareMathOperator{\epi}{epi}
\DeclareMathOperator{\inte}{int}
\DeclareMathOperator{\ri}{ri}
\DeclareMathOperator*{\argmin}{argmin}
\DeclareMathOperator*{\argmax}{argmax}
\DeclareMathOperator{\graph}{graph}
\DeclareMathOperator{\sgn}{sgn}
\DeclareMathOperator*{\Rep}{Rep}
\DeclareMathOperator{\Proj}{Proj}
\DeclareMathOperator{\prox}{prox}
\DeclareMathOperator{\mat}{mat}
\let\vec\relax
\DeclareMathOperator{\vec}{vec}
\let\Re\relax
\DeclareMathOperator{\Re}{Re}
\let\Im\relax
\DeclareMathOperator{\Im}{Im}
% Put x \to \infty below \lim
\let\svlim\lim\def\lim{\svlim\limits}

%wide hat
\usepackage{scalerel,stackengine}
\stackMath
\newcommand*\wh[1]{%
\savestack{\tmpbox}{\stretchto{%
  \scaleto{%
    \scalerel*[\widthof{\ensuremath{#1}}]{\kern-.6pt\bigwedge\kern-.6pt}%
    {\rule[-\textheight/2]{1ex}{\textheight}}%WIDTH-LIMITED BIG WEDGE
  }{\textheight}% 
}{0.5ex}}%
\stackon[1pt]{#1}{\tmpbox}%
}
\parskip 1ex

%Make implies and impliedby shorter
\let\implies\Rightarrow
\let\impliedby\Leftarrow
\let\iff\Leftrightarrow
\let\epsilon\varepsilon

% Add \contra symbol to denote contradiction
\usepackage{stmaryrd} % for \lightning
\newcommand\contra{\scalebox{1.5}{$\lightning$}}

% \let\phi\varphi

% Command for short corrections
% Usage: 1+1=\correct{3}{2}

\definecolor{correct}{HTML}{009900}
\newcommand\correct[2]{\ensuremath{\:}{\color{red}{#1}}\ensuremath{\to }{\color{correct}{#2}}\ensuremath{\:}}
\newcommand\green[1]{{\color{correct}{#1}}}

% horizontal rule
\newcommand\hr{
    \noindent\rule[0.5ex]{\linewidth}{0.5pt}
}

% hide parts
\newcommand\hide[1]{}

% si unitx
\usepackage{siunitx}
\sisetup{locale = FR}

%allows pmatrix to stretch
\makeatletter
\renewcommand*\env@matrix[1][\arraystretch]{%
  \edef\arraystretch{#1}%
  \hskip -\arraycolsep
  \let\@ifnextchar\new@ifnextchar
  \array{*\c@MaxMatrixCols c}}
\makeatother

\renewcommand{\arraystretch}{0.8}

% Environments
\makeatother
% For box around Definition, Theorem, \ldots
%%fakesection Theorems
\usepackage{thmtools}
\usepackage[framemethod=TikZ]{mdframed}

\theoremstyle{definition}
\mdfdefinestyle{mdbluebox}{%
	roundcorner = 10pt,
	linewidth=1pt,
	skipabove=12pt,
	innerbottommargin=9pt,
	skipbelow=2pt,
	nobreak=true,
	linecolor=blue,
	backgroundcolor=TealBlue!5,
}
\declaretheoremstyle[
	headfont=\sffamily\bfseries\color{MidnightBlue},
	mdframed={style=mdbluebox},
	headpunct={\\[3pt]},
	postheadspace={0pt}
]{thmbluebox}

\mdfdefinestyle{mdredbox}{%
	linewidth=0.5pt,
	skipabove=12pt,
	frametitleaboveskip=5pt,
	frametitlebelowskip=0pt,
	skipbelow=2pt,
	frametitlefont=\bfseries,
	innertopmargin=4pt,
	innerbottommargin=8pt,
	nobreak=false,
	linecolor=RawSienna,
	backgroundcolor=Salmon!5,
}
\declaretheoremstyle[
	headfont=\bfseries\color{RawSienna},
	mdframed={style=mdredbox},
	headpunct={\\[3pt]},
	postheadspace={0pt},
]{thmredbox}

\declaretheorem[%
style=thmbluebox,name=Theorem,numberwithin=section]{thm}
\declaretheorem[style=thmbluebox,name=Lemma,sibling=thm]{lem}
\declaretheorem[style=thmbluebox,name=Proposition,sibling=thm]{prop}
\declaretheorem[style=thmbluebox,name=Corollary,sibling=thm]{coro}
\declaretheorem[style=thmredbox,name=Example,sibling=thm]{eg}

\mdfdefinestyle{mdgreenbox}{%
	roundcorner = 10pt,
	linewidth=1pt,
	skipabove=12pt,
	innerbottommargin=9pt,
	skipbelow=2pt,
	nobreak=true,
	linecolor=ForestGreen,
	backgroundcolor=ForestGreen!5,
}

\declaretheoremstyle[
	headfont=\bfseries\sffamily\color{ForestGreen!70!black},
	bodyfont=\normalfont,
	spaceabove=2pt,
	spacebelow=1pt,
	mdframed={style=mdgreenbox},
	headpunct={ --- },
]{thmgreenbox}

\declaretheorem[style=thmgreenbox,name=Definition,sibling=thm]{defn}

\mdfdefinestyle{mdgreenboxsq}{%
	linewidth=1pt,
	skipabove=12pt,
	innerbottommargin=9pt,
	skipbelow=2pt,
	nobreak=true,
	linecolor=ForestGreen,
	backgroundcolor=ForestGreen!5,
}
\declaretheoremstyle[
	headfont=\bfseries\sffamily\color{ForestGreen!70!black},
	bodyfont=\normalfont,
	spaceabove=2pt,
	spacebelow=1pt,
	mdframed={style=mdgreenboxsq},
	headpunct={},
]{thmgreenboxsq}
\declaretheoremstyle[
	headfont=\bfseries\sffamily\color{ForestGreen!70!black},
	bodyfont=\normalfont,
	spaceabove=2pt,
	spacebelow=1pt,
	mdframed={style=mdgreenboxsq},
	headpunct={},
]{thmgreenboxsq*}

\mdfdefinestyle{mdblackbox}{%
	skipabove=8pt,
	linewidth=3pt,
	rightline=false,
	leftline=true,
	topline=false,
	bottomline=false,
	linecolor=black,
	backgroundcolor=RedViolet!5!gray!5,
}
\declaretheoremstyle[
	headfont=\bfseries,
	bodyfont=\normalfont\small,
	spaceabove=0pt,
	spacebelow=0pt,
	mdframed={style=mdblackbox}
]{thmblackbox}

\theoremstyle{plain}
\declaretheorem[name=Question,sibling=thm,style=thmblackbox]{ques}
\declaretheorem[name=Remark,sibling=thm,style=thmgreenboxsq]{remark}
\declaretheorem[name=Remark,sibling=thm,style=thmgreenboxsq*]{remark*}

\theoremstyle{definition}
\newtheorem{claim}[thm]{Claim}
\theoremstyle{remark}
\newtheorem*{case}{Case}
\newtheorem*{notation}{Notation}
\newtheorem*{note}{Note}
\newtheorem*{motivation}{Motivation}
\newtheorem*{intuition}{Intuition}

% Make section starts with 1 for report type
%\renewcommand\thesection{\arabic{section}}

% End example and intermezzo environments with a small diamond (just like proof
% environments end with a small square)
\usepackage{etoolbox}
\AtEndEnvironment{vb}{\null\hfill$\diamond$}%
\AtEndEnvironment{intermezzo}{\null\hfill$\diamond$}%
% \AtEndEnvironment{opmerking}{\null\hfill$\diamond$}%

% Fix some spacing
% http://tex.stackexchange.com/questions/22119/how-can-i-change-the-spacing-before-theorems-with-amsthm
\makeatletter
\def\thm@space@setup{%
  \thm@preskip=\parskip \thm@postskip=0pt
}

% Fix some stuff
% %http://tex.stackexchange.com/questions/76273/multiple-pdfs-with-page-group-included-in-a-single-page-warning
\pdfsuppresswarningpagegroup=1

\renewcommand{\baselinestretch}{1.5}
\RequirePackage{hyperref}[6.83]
\hypersetup{
  colorlinks=false,
  frenchlinks=false,
  pdfborder={0 0 0},
  naturalnames=false,
  hypertexnames=false,
  breaklinks
}
\urlstyle{same}

\usepackage{graphics}
\usepackage{epstopdf}

%%
%% Add support for color in order to color the hyperlinks.
%% 
\hypersetup{
  colorlinks = true,
  allcolors = siaminlinkcolor,
  urlcolor = siamexlinkcolor,
}
%%fakesection Links
\hypersetup{
    colorlinks,
    linkcolor={red!50!black},
    citecolor={green!50!black},
    urlcolor={blue!80!black}
}
%customization of cleveref
\RequirePackage[capitalize,nameinlink]{cleveref}[0.19]

% Per SIAM Style Manual, "section" should be lowercase
\crefname{section}{section}{sections}
\crefname{subsection}{subsection}{subsections}
\Crefname{section}{Section}{Sections}
\Crefname{subsection}{Subsection}{Subsections}

% Per SIAM Style Manual, "Figure" should be spelled out in references
\Crefname{figure}{Figure}{Figures}

% Per SIAM Style Manual, don't say equation in front on an equation.
\crefformat{equation}{\textup{#2(#1)#3}}
\crefrangeformat{equation}{\textup{#3(#1)#4--#5(#2)#6}}
\crefmultiformat{equation}{\textup{#2(#1)#3}}{ and \textup{#2(#1)#3}}
{, \textup{#2(#1)#3}}{, and \textup{#2(#1)#3}}
\crefrangemultiformat{equation}{\textup{#3(#1)#4--#5(#2)#6}}%
{ and \textup{#3(#1)#4--#5(#2)#6}}{, \textup{#3(#1)#4--#5(#2)#6}}{, and \textup{#3(#1)#4--#5(#2)#6}}

% But spell it out at the beginning of a sentence.
\Crefformat{equation}{#2Equation~\textup{(#1)}#3}
\Crefrangeformat{equation}{Equations~\textup{#3(#1)#4--#5(#2)#6}}
\Crefmultiformat{equation}{Equations~\textup{#2(#1)#3}}{ and \textup{#2(#1)#3}}
{, \textup{#2(#1)#3}}{, and \textup{#2(#1)#3}}
\Crefrangemultiformat{equation}{Equations~\textup{#3(#1)#4--#5(#2)#6}}%
{ and \textup{#3(#1)#4--#5(#2)#6}}{, \textup{#3(#1)#4--#5(#2)#6}}{, and \textup{#3(#1)#4--#5(#2)#6}}

% Make number non-italic in any environment.
\crefdefaultlabelformat{#2\textup{#1}#3}

% My name
\author{Jaden Wang}



\begin{document}
\newpage
\chapter{Algorithms}
\newpage
\section{Unconstrained Optimization}
We assume reasonable smoothness of the objective. Here is an overview of the algorithms:
\begin{enumerate}[label=(\arabic*)]
	\item gradient descent
		\[
			x_{k+1} = x_k - t \cdot  \nabla f(x_k), \quad t= \frac{1}{L}
		.\] 
		where $ L$ is the Lipschitz constant of the gradient.
	\item Newton's method
		\[
			x_{k+1} = x_k - \left( \nabla ^2 f(x_k) \right)^{-1} \nabla f(x_k)
		.\] 
		This can reduce to gradient descent when we have $ \nabla ^2 f(x) \preceq L \cdot I$ and we just bound the Hessian with $ L \cdot I$.
	\item Quasi-Newton
\end{enumerate}
\newpage
\subsection{Proximal gradient descent}
\begin{align*}
	\min \underbrace{ f(x)}_{ \text{ smooth, strongly convex} } + \underbrace{ g(x)}_{ \text{simple, convex}}  
\end{align*}
Note we can add indicator function to $ g$:
\[
	g(x) = I_{C}(x) + h(x)
,\] 
\emph{i.e.} when we have constraint $ x \in C$.
\subsubsection{motivation}
We can try the first-order Taylor approximation of $ f$. However, recall minimizing a linear function would go to negative infinity, so we need to go to 2nd order.
\begin{align*}
	x_{k+1} &= \argmin_{x} f(x_k) + \langle \nabla f(x_k),x-x_k \rangle + \frac{1}{2} L\norm{ x-x_k}_2^2 + g(x)\\
	&= \argmin_x \frac{1}{L} \left(  \langle \nabla f(x_k),x-x_k \rangle + \frac{1}{2} L\norm{ x-x_k}_2^2 + g(x)\right)  \\
	&= \argmin_x \frac{1}{2} \norm{ x-\left(x_k - \frac{1}{L} \nabla f(x_k)\right)}_2^2 + \frac{1}{L} \cdot g(x) \text{ complete the square and ignore constants}  \\
	&= \argmin_x \frac{1}{2} \norm{ x-\widetilde{ x}}_2^2 + \frac{1}{L} \cdot g(x)  \\
	&= \prox_{\frac{1}{L} \cdot g} (\widetilde{ x})
\end{align*}
Note that this solution is unique because we have strong convexity.
\subsubsection{algorithm}
\begin{align*}
	x_{k+1} = \prox_{t g} (x_k - t \cdot \nabla f(x_k))\qquad t\text{ via line search or } t=\frac{1}{L} 
\end{align*}
\begin{remark}
	If $ g(x)=0$, proximal operator is the identity function so it reduces to gradient descent.
\end{remark}

\begin{eg}
	$ g(x) = I_C$. Then
	 \begin{align*}
		 \prox_{t g} (\widetilde{ x}) = \Proj_C(x)
	\end{align*}
	Recall from linear algebra: if $ \Proj_V(\widetilde{ x})$ is the projection of $ \widetilde{ x} \to V$, then
	\[
		\widetilde{ x} =\Proj_V(\widetilde{ x}) + \Proj_{V^{\perp}}(\widetilde{ x})
	.\]
	We can generalize this result to \allbold{Moreau's decomposition}: 
	\begin{align*}
		\widetilde{ x} = \prox_g (\widetilde{ x}) + \prox_{g^* } (\widetilde{ x})
	\end{align*}
\end{eg}
	\begin{eg}
	\[
		\Proj_{\norm{ x}_{\infty} \leq 1} = \widetilde{ x} - \prox_{\norm{ \cdot }_1 } (\widetilde{ x})
	.\] 

	\begin{align*}
		\prox_{t \norm{ \cdot }_1 } (y) = \argmin_x \frac{1}{2} \norm{ x-y}_2 ^2 + L \norm{ x}_1 \text{ this is separable!}
	\end{align*}
	By Fermat's rule,
	\begin{align*}
		\prox_g (y) = \argmin \frac{1}{2}\norm{ x-y}^2 + g(x)\\
		\implies 0 &\in x-y + \partial g(x)\\
		y&\in x + \partial g(x)  \\
		y & \in (I+\partial g)(x)  \\
		x&\in \left( I+ \partial g \right)^{-1} (y)   \\
		x&= \left( I+ \partial \norm{ }_1  \right)^{-1} y  \text{ unique solution s.c.}
	\end{align*}
	We derived earlier that the solution to $ \prox_{t \cdot \norm{ \cdot }_1 }$ is
	\[
		x= \sgn(y) \cdot \lfloor |y| - t\rfloor_{+}
	.\] 
	\end{eg}

\subsubsection{alternative derivation}
By Fermat
\begin{align*}
	0 &\in \partial (f+g)(x)  \\
	0 &= \in \nabla  f(x) + \partial g(x) \text{ under CQ} \\
	x &= x +  \nabla  f(x) + \partial g(x) \\
	x-\nabla f(x) &\in x + \partial g(x) = (I + \partial g)(x)  \\
	x &= \left( I+ \partial g \right) ^{-1} (I - \nabla f)(x) \text{ fixed point eqn} \\
	x_{k+1} &= \left( I+ \partial g \right) ^{-1} (I - \nabla f)(x_k)\\
		&= \prox_g (x_k - \nabla f(x_k)) 
\end{align*}
If $ f = 0$, we get
 \begin{align*}
	 x_{k+1} &= \prox_{t g}(x_k) \text{ here t is anything we want since }f=0 \\
		 &= \argmin t \cdot g(x) + \frac{1}{2} \norm{ x-x_k}^2  \\
\end{align*}
\begin{remark}
Forward Euler exactly corresponds to gradient descent, whereas backward Euler exactly corresponds to proximal gradient descent. Thus, proximal gradient descent is also called "forward-backward method".
\end{remark}
\end{document}
