\documentclass[class=article,crop=false]{standalone} 
%Fall 2020
% Some basic packages
\usepackage{standalone}[subpreambles=true]
\usepackage[utf8]{inputenc}
\usepackage[T1]{fontenc}
\usepackage{textcomp}
\usepackage[english]{babel}
\usepackage{url}
\usepackage{graphicx}
\usepackage{float}
\usepackage{enumitem}
\usepackage{lmodern}
\usepackage{hyperref}
\usepackage[usenames,svgnames,dvipsnames]{xcolor}


\pdfminorversion=7

% Don't indent paragraphs, leave some space between them
\usepackage{parskip}

% Hide page number when page is empty
\usepackage{emptypage}
\usepackage{subcaption}
\usepackage{multicol}
\usepackage[dvipsnames]{xcolor}
\usepackage[b]{esvect}

% Math stuff
\usepackage{amsmath, amsfonts, mathtools, amsthm, amssymb}
\usepackage{bbm}

% Fancy script capitals
\usepackage{mathrsfs}
\usepackage{cancel}
% Bold math
\usepackage{bm}
% Some shortcuts
\newcommand{\rr}{\ensuremath{\mathbb{R}}}
\newcommand{\zz}{\ensuremath{\mathbb{Z}}}
\newcommand{\qq}{\ensuremath{\mathbb{Q}}}
\newcommand{\nn}{\ensuremath{\mathbb{N}}}
\newcommand{\ff}{\ensuremath{\mathbb{F}}}
\newcommand{\cc}{\ensuremath{\mathbb{C}}}
\newcommand{\ee}{\ensuremath{\mathbb{E}}}
\renewcommand\O{\ensuremath{\emptyset}}
\newcommand{\norm}[1]{{\left\lVert{#1}\right\rVert}}
\newcommand{\ve}[1]{{\mathbf{#1}}}
\newcommand\allbold[1]{{\boldmath\textbf{#1}}}
\DeclareMathOperator{\lcm}{lcm}
\DeclareMathOperator{\im}{im}
\DeclareMathOperator{\coim}{coim}
\DeclareMathOperator{\dom}{dom}
\DeclareMathOperator{\tr}{tr}
\DeclareMathOperator{\rank}{rank}
\DeclareMathOperator*{\var}{Var}
\DeclareMathOperator*{\ev}{E}
\DeclareMathOperator{\sinc}{sinc}
\DeclareMathOperator{\dg}{deg}
\DeclareMathOperator{\aff}{aff}
\DeclareMathOperator{\conv}{conv}
\DeclareMathOperator{\epi}{epi}
\DeclareMathOperator{\inte}{int}
\DeclareMathOperator{\ri}{ri}
\DeclareMathOperator*{\argmin}{argmin}
\DeclareMathOperator*{\argmax}{argmax}
\DeclareMathOperator{\graph}{graph}
\DeclareMathOperator{\sgn}{sgn}
\DeclareMathOperator*{\Rep}{Rep}
\DeclareMathOperator{\Proj}{Proj}
\DeclareMathOperator{\prox}{prox}
\DeclareMathOperator{\mat}{mat}
\let\vec\relax
\DeclareMathOperator{\vec}{vec}
\let\Re\relax
\DeclareMathOperator{\Re}{Re}
\let\Im\relax
\DeclareMathOperator{\Im}{Im}
% Put x \to \infty below \lim
\let\svlim\lim\def\lim{\svlim\limits}

%wide hat
\usepackage{scalerel,stackengine}
\stackMath
\newcommand*\wh[1]{%
\savestack{\tmpbox}{\stretchto{%
  \scaleto{%
    \scalerel*[\widthof{\ensuremath{#1}}]{\kern-.6pt\bigwedge\kern-.6pt}%
    {\rule[-\textheight/2]{1ex}{\textheight}}%WIDTH-LIMITED BIG WEDGE
  }{\textheight}% 
}{0.5ex}}%
\stackon[1pt]{#1}{\tmpbox}%
}
\parskip 1ex

%Make implies and impliedby shorter
\let\implies\Rightarrow
\let\impliedby\Leftarrow
\let\iff\Leftrightarrow
\let\epsilon\varepsilon

% Add \contra symbol to denote contradiction
\usepackage{stmaryrd} % for \lightning
\newcommand\contra{\scalebox{1.5}{$\lightning$}}

% \let\phi\varphi

% Command for short corrections
% Usage: 1+1=\correct{3}{2}

\definecolor{correct}{HTML}{009900}
\newcommand\correct[2]{\ensuremath{\:}{\color{red}{#1}}\ensuremath{\to }{\color{correct}{#2}}\ensuremath{\:}}
\newcommand\green[1]{{\color{correct}{#1}}}

% horizontal rule
\newcommand\hr{
    \noindent\rule[0.5ex]{\linewidth}{0.5pt}
}

% hide parts
\newcommand\hide[1]{}

% si unitx
\usepackage{siunitx}
\sisetup{locale = FR}

%allows pmatrix to stretch
\makeatletter
\renewcommand*\env@matrix[1][\arraystretch]{%
  \edef\arraystretch{#1}%
  \hskip -\arraycolsep
  \let\@ifnextchar\new@ifnextchar
  \array{*\c@MaxMatrixCols c}}
\makeatother

\renewcommand{\arraystretch}{0.8}

% Environments
\makeatother
% For box around Definition, Theorem, \ldots
%%fakesection Theorems
\usepackage{thmtools}
\usepackage[framemethod=TikZ]{mdframed}

\theoremstyle{definition}
\mdfdefinestyle{mdbluebox}{%
	roundcorner = 10pt,
	linewidth=1pt,
	skipabove=12pt,
	innerbottommargin=9pt,
	skipbelow=2pt,
	nobreak=true,
	linecolor=blue,
	backgroundcolor=TealBlue!5,
}
\declaretheoremstyle[
	headfont=\sffamily\bfseries\color{MidnightBlue},
	mdframed={style=mdbluebox},
	headpunct={\\[3pt]},
	postheadspace={0pt}
]{thmbluebox}

\mdfdefinestyle{mdredbox}{%
	linewidth=0.5pt,
	skipabove=12pt,
	frametitleaboveskip=5pt,
	frametitlebelowskip=0pt,
	skipbelow=2pt,
	frametitlefont=\bfseries,
	innertopmargin=4pt,
	innerbottommargin=8pt,
	nobreak=false,
	linecolor=RawSienna,
	backgroundcolor=Salmon!5,
}
\declaretheoremstyle[
	headfont=\bfseries\color{RawSienna},
	mdframed={style=mdredbox},
	headpunct={\\[3pt]},
	postheadspace={0pt},
]{thmredbox}

\declaretheorem[%
style=thmbluebox,name=Theorem,numberwithin=section]{thm}
\declaretheorem[style=thmbluebox,name=Lemma,sibling=thm]{lem}
\declaretheorem[style=thmbluebox,name=Proposition,sibling=thm]{prop}
\declaretheorem[style=thmbluebox,name=Corollary,sibling=thm]{coro}
\declaretheorem[style=thmredbox,name=Example,sibling=thm]{eg}

\mdfdefinestyle{mdgreenbox}{%
	roundcorner = 10pt,
	linewidth=1pt,
	skipabove=12pt,
	innerbottommargin=9pt,
	skipbelow=2pt,
	nobreak=true,
	linecolor=ForestGreen,
	backgroundcolor=ForestGreen!5,
}

\declaretheoremstyle[
	headfont=\bfseries\sffamily\color{ForestGreen!70!black},
	bodyfont=\normalfont,
	spaceabove=2pt,
	spacebelow=1pt,
	mdframed={style=mdgreenbox},
	headpunct={ --- },
]{thmgreenbox}

\declaretheorem[style=thmgreenbox,name=Definition,sibling=thm]{defn}

\mdfdefinestyle{mdgreenboxsq}{%
	linewidth=1pt,
	skipabove=12pt,
	innerbottommargin=9pt,
	skipbelow=2pt,
	nobreak=true,
	linecolor=ForestGreen,
	backgroundcolor=ForestGreen!5,
}
\declaretheoremstyle[
	headfont=\bfseries\sffamily\color{ForestGreen!70!black},
	bodyfont=\normalfont,
	spaceabove=2pt,
	spacebelow=1pt,
	mdframed={style=mdgreenboxsq},
	headpunct={},
]{thmgreenboxsq}
\declaretheoremstyle[
	headfont=\bfseries\sffamily\color{ForestGreen!70!black},
	bodyfont=\normalfont,
	spaceabove=2pt,
	spacebelow=1pt,
	mdframed={style=mdgreenboxsq},
	headpunct={},
]{thmgreenboxsq*}

\mdfdefinestyle{mdblackbox}{%
	skipabove=8pt,
	linewidth=3pt,
	rightline=false,
	leftline=true,
	topline=false,
	bottomline=false,
	linecolor=black,
	backgroundcolor=RedViolet!5!gray!5,
}
\declaretheoremstyle[
	headfont=\bfseries,
	bodyfont=\normalfont\small,
	spaceabove=0pt,
	spacebelow=0pt,
	mdframed={style=mdblackbox}
]{thmblackbox}

\theoremstyle{plain}
\declaretheorem[name=Question,sibling=thm,style=thmblackbox]{ques}
\declaretheorem[name=Remark,sibling=thm,style=thmgreenboxsq]{remark}
\declaretheorem[name=Remark,sibling=thm,style=thmgreenboxsq*]{remark*}

\theoremstyle{definition}
\newtheorem{claim}[thm]{Claim}
\theoremstyle{remark}
\newtheorem*{case}{Case}
\newtheorem*{notation}{Notation}
\newtheorem*{note}{Note}
\newtheorem*{motivation}{Motivation}
\newtheorem*{intuition}{Intuition}

% Make section starts with 1 for report type
%\renewcommand\thesection{\arabic{section}}

% End example and intermezzo environments with a small diamond (just like proof
% environments end with a small square)
\usepackage{etoolbox}
\AtEndEnvironment{vb}{\null\hfill$\diamond$}%
\AtEndEnvironment{intermezzo}{\null\hfill$\diamond$}%
% \AtEndEnvironment{opmerking}{\null\hfill$\diamond$}%

% Fix some spacing
% http://tex.stackexchange.com/questions/22119/how-can-i-change-the-spacing-before-theorems-with-amsthm
\makeatletter
\def\thm@space@setup{%
  \thm@preskip=\parskip \thm@postskip=0pt
}

% Fix some stuff
% %http://tex.stackexchange.com/questions/76273/multiple-pdfs-with-page-group-included-in-a-single-page-warning
\pdfsuppresswarningpagegroup=1

\renewcommand{\baselinestretch}{1.5}
\RequirePackage{hyperref}[6.83]
\hypersetup{
  colorlinks=false,
  frenchlinks=false,
  pdfborder={0 0 0},
  naturalnames=false,
  hypertexnames=false,
  breaklinks
}
\urlstyle{same}

\usepackage{graphics}
\usepackage{epstopdf}

%%
%% Add support for color in order to color the hyperlinks.
%% 
\hypersetup{
  colorlinks = true,
  allcolors = siaminlinkcolor,
  urlcolor = siamexlinkcolor,
}
%%fakesection Links
\hypersetup{
    colorlinks,
    linkcolor={red!50!black},
    citecolor={green!50!black},
    urlcolor={blue!80!black}
}
%customization of cleveref
\RequirePackage[capitalize,nameinlink]{cleveref}[0.19]

% Per SIAM Style Manual, "section" should be lowercase
\crefname{section}{section}{sections}
\crefname{subsection}{subsection}{subsections}
\Crefname{section}{Section}{Sections}
\Crefname{subsection}{Subsection}{Subsections}

% Per SIAM Style Manual, "Figure" should be spelled out in references
\Crefname{figure}{Figure}{Figures}

% Per SIAM Style Manual, don't say equation in front on an equation.
\crefformat{equation}{\textup{#2(#1)#3}}
\crefrangeformat{equation}{\textup{#3(#1)#4--#5(#2)#6}}
\crefmultiformat{equation}{\textup{#2(#1)#3}}{ and \textup{#2(#1)#3}}
{, \textup{#2(#1)#3}}{, and \textup{#2(#1)#3}}
\crefrangemultiformat{equation}{\textup{#3(#1)#4--#5(#2)#6}}%
{ and \textup{#3(#1)#4--#5(#2)#6}}{, \textup{#3(#1)#4--#5(#2)#6}}{, and \textup{#3(#1)#4--#5(#2)#6}}

% But spell it out at the beginning of a sentence.
\Crefformat{equation}{#2Equation~\textup{(#1)}#3}
\Crefrangeformat{equation}{Equations~\textup{#3(#1)#4--#5(#2)#6}}
\Crefmultiformat{equation}{Equations~\textup{#2(#1)#3}}{ and \textup{#2(#1)#3}}
{, \textup{#2(#1)#3}}{, and \textup{#2(#1)#3}}
\Crefrangemultiformat{equation}{Equations~\textup{#3(#1)#4--#5(#2)#6}}%
{ and \textup{#3(#1)#4--#5(#2)#6}}{, \textup{#3(#1)#4--#5(#2)#6}}{, and \textup{#3(#1)#4--#5(#2)#6}}

% Make number non-italic in any environment.
\crefdefaultlabelformat{#2\textup{#1}#3}

% My name
\author{Jaden Wang}



\begin{document}

\begin{remark}
Euler inequality is an example of a \allbold{variational inequality}, where we can replace the gradient with any operator.
\end{remark}

\subsubsection{unconstrained problems}
	$ C= \rr^{n}, dom(f) = \rr^{n}$. Choose $ y = x- \nabla f(x)$, we have
	\[
		-\norm{ \nabla f(x)} ^2 \geq 0 \iff \nabla f(x) = 0
	,\]
	and we recover Fermat.

\subsubsection{convex equality/affine constraints}
	$ C = \{x: Ax=b\}$ where $ A \in \rr^{m} \times \rr^{n}$. Then (2) becomes for all $ y$  s.t. $ Ay=b$,  $ \langle \nabla f(x),y-x \rangle\geq 0$. We can write $ y= y_p +v, v \in \ker(A)$. Since $ Ax = b$, we can take $ y_p = x$ so $ y=x+v$. Then it becomes $ \ \forall \ v \in \ker A$, $ \langle \nabla f(x),v  \rangle \geq 0$. Since $ \ker A$ is closed under negation, we need $ \langle \nabla f(x),-v \rangle\geq 0$. Thus we get the necessary and sufficient condition for $ x$ to be optimal:
	\[
		\langle \nabla f(x),v \rangle = 0
	.\]
	This means $ \nabla f(x) \perp \ker A$. Recall that the coimage space is orthogonal to the kernel, $ \nabla f(x) \in \coim A = \im A^{T}$. Thus the condition is equivalent to there exists a $ \nu \in \rr^{m}$ s.t. $ \nabla f(x) = A^{T} (-\nu)$. Rearranging yields: 
	\[
		\nabla f(x) + A^{T}\nu = 0
	.\] 
The $ \nu$ here is the \allbold{Lagrange multipliers}. 
\subsection{Conic optimization problems [BV04 Ch.4.3, 4.4, 4.6]}

\subsubsection{Linear programs}
Linear objective with polyhedra constraints.

Standard forms:
\begin{alignat*}{3}
	\min\ & \langle c,x \rangle \qquad\qquad  \min\ &&\langle c,x \rangle \qquad\qquad \min\ &&\langle c,x \rangle\\
	\text{subject to } &Gx \leq h &&x\geq 0 \qquad  &&Ax\leq b\\
			   &Ax = b &&Ax=b &&
\end{alignat*}	
\begin{eg}[convert first to second form]
Use slack variable $ s\geq 0$  s.t. $ Gx+s =b$, $ x=x^{+}-x^{-}, x^{+},x^{-}\geq 0$. Let $ \widetilde{ x} = \begin{pmatrix} x^{+}\\x^{-}\\s \end{pmatrix} \geq 0$. Then we can combine the equality constraints into $ \widetilde{ A}\widetilde{ x}  = \widetilde{ b}$.
\end{eg}

\begin{eg}
$ \min_x \norm{ x}_1 $ s.t. $ Ax=b$. Since this is separable/piecewise-linear, we use slack variables to rewrite $ x_i = x_i^{+} - x_i^{-}, x_i^{+}, x_i^{-}\geq 0$. Thus, $ |x_i| \leq x_i^{+}+ x_i^{-}$ which will be equality when we minimize. Now we have an equivalent problem
\begin{align*}
	\min_{x^{+}, x^{-}}\ & \mathbbm{1}^{T}(x^{+}+ x^{-}) \\
\text{subject to } &Ax^{+}-Ax^{-} =b
\end{align*}
\end{eg}
\begin{eg}[epigraph trick]
\begin{align*}
\min_x\ & \norm{ x}_{\infty}  \\
\text{subject to } &Ax=b
\end{align*}
is equivalent to
\begin{align*}
	\min_{x,t}\ & t\\
	\text{subject to } &Ax=b\\
		&\norm{ x}_{\infty} \leq t
\end{align*}
\end{eg}
\begin{remark}
	LP is the most well-studied kind of convex optimization problem. It has wide applications in industry and can handle very large dimensions. However, integer LP is not convex and is NP-hard. But it is not impossible. In fact, it just takes a little longer to run and we can still prove that we have the best solution when we find one. CPLEX, MOSEK, Gurobi\ldots 
\end{remark}


\begin{thm}
LP solution set always includes a vertex of the polyhedron.
\end{thm}

\subsubsection{Quadratic programs}

\begin{align*}
\min_x\ & \frac{1}{2} \langle x,Px \rangle + \langle q,x \rangle + r \\
\text{subject to } &Gx \leq h \\
&Ax = b
\end{align*}
\begin{note}
Quadratic programs are not always convex. It must have $ P \succeq 0$ to be convex.
\end{note}

\begin{eg}[Regression]
 \begin{align*}
	\min \frac{1}{2} \norm{ Ax-b}^2 &= \frac{1}{2} \langle x,\underbrace{ A^{T}A}_{ \succeq 0} x \rangle - \langle x, A^{T}b \rangle + \norm{ b}^2
\end{align*}
\end{eg}
\subsubsection{Quadratically constrained quadratic program (QCQP)}
\begin{align*}
\min_x\ & \frac{1}{2} \langle x,Px \rangle + \langle q,x \rangle + r \\
\text{subject to } &\frac{1}{2} \langle x,P_i x \rangle + \langle q_i,x \rangle + r_i  \leq 0 \ \forall \ i=1,\ldots,m\\
&Ax = b
\end{align*}
\subsubsection{Second-order cone program (SOCP)}
It generalizes convex QCQP:
\begin{align*}
\min_x\ & \langle c_0,x \rangle \\
\text{subject to } &\norm{ A_i x+b_i}_2 \leq \langle c_i,x \rangle+ d_i\\
&Fx = g
\end{align*}
The inequality constraint is an affine composition with 2nd order cone (recall $ (y,t) \in \mathcal{ K} \iff \norm{ y}_2 \leq t$). This can be solved efficiently via cvxpy.
\subsubsection{Conic programming}
The standard form (2nd form) of LP is an example of conic program where the proper cone $ \mathcal{ K} = \rr_{+}^{n}$. More generally, a \allbold{conic program} is
\begin{alignat*}{2}
	\min\ & \langle c,x \rangle \qquad\qquad \qquad   \min\ &&\langle c,x \rangle \\
	\text{subject to } &Fx +g \preceq_{ \mathcal{ K}} 0 &&x \succeq_{ \mathcal{ K}} 0 \\
			   &Ax = b &&Ax=b
\end{alignat*}	
where $ \mathcal{ K}$ is a proper cone (closed, convex, solid, pointed) and $ y \succeq _{ \mathcal{ K}}0$ means $ y \in \mathcal{ K}$.

Recall that if $ \mathcal{ K}_1, \mathcal{ K}_2$ are proper cones, so is $ \mathcal{ K} := \mathcal{ K}_1 \times \mathcal{ K}_2$. Thus for two different inequality constraints in the first form, we can instead write
\[
\begin{pmatrix} F_1\\F_2 \end{pmatrix} x + \begin{pmatrix} g_1\\g_2 \end{pmatrix} \preceq_{ \mathcal{ K}} 0
.\] 
This is how SOCPs is a conic program.
\subsubsection{Semi-definite programs}
This is a powerful tool that revolutionized engineering. Here the proper cone $ \mathcal{ K} = \rr_{+}^{n}$ (and direct products of these).
\begin{notation}
	$ \mathcal{ K}$ is omitted in $ \succeq$ here because SDP is so common.
\end{notation}
\begin{align*}
	\min\ & \langle C,X \rangle \\
	\text{subject to } & \langle A_i,X \rangle =b_i, i = 1,\ldots,m \iff \mathcal{ A}(X) = b\\
&X \succeq 0 
\end{align*}
\begin{note}
Here $ \mathcal{ A}$ is some abstract linear operator that can be flatten to a $ m \times n^2$ matrix if we vectorize $ X$.
\end{note}
\begin{remark}
	$ \langle C,X \rangle = \tr\left( C^{T} X \right) = \langle \vec{C},\vec{X} \rangle$. To implement this, we do not want to multiple the matrices. Since $ \rr^{ n \times n} \cong \rr^{n^2}$, we can vectorize it and achieves $ \mathcal{ O}(n^2)$. 
\end{remark}
\begin{remark}
$ \vec: \rr^{ n \times n} \to \rr^{n^2}$ and $ \mat : \rr^{n^2} \to \rr^{n \times n}$ are inverses.

$ S^{n} \cong \rr^{ n(n+1) /2}$ because we can remove redundancy from symmetry. Although if memory is not an issue, it is usually easier to just use the whole thing since we have to reweight off-diagonal terms.
\end{remark}

\begin{remark}
	We can treat a matrix as a vector using trace for inner product and Frobenius norm or as a transformation using spectral norm (doesn't have inner product because it's in Banach space not in Hilbert space). 
\end{remark}

\begin{remark}
We have the Kronecker product trick
\[
	\vec(A X B) = \left( B^{T} \otimes A \right) \vec(X)
.\] 
Not all linear operator on matrices $ \mathcal{ A}(X)$ are in this form because we need to factorize the operator. Moreover, LHS is computationally cheaper than RHS.
\end{remark}
\end{document}
