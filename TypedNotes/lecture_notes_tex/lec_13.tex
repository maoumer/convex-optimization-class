\documentclass[class=article,crop=false]{standalone} 
%Fall 2020
% Some basic packages
\usepackage{standalone}[subpreambles=true]
\usepackage[utf8]{inputenc}
\usepackage[T1]{fontenc}
\usepackage{textcomp}
\usepackage[english]{babel}
\usepackage{url}
\usepackage{graphicx}
\usepackage{float}
\usepackage{enumitem}
\usepackage{lmodern}
\usepackage{hyperref}
\usepackage[usenames,svgnames,dvipsnames]{xcolor}


\pdfminorversion=7

% Don't indent paragraphs, leave some space between them
\usepackage{parskip}

% Hide page number when page is empty
\usepackage{emptypage}
\usepackage{subcaption}
\usepackage{multicol}
\usepackage[dvipsnames]{xcolor}
\usepackage[b]{esvect}

% Math stuff
\usepackage{amsmath, amsfonts, mathtools, amsthm, amssymb}
\usepackage{bbm}

% Fancy script capitals
\usepackage{mathrsfs}
\usepackage{cancel}
% Bold math
\usepackage{bm}
% Some shortcuts
\newcommand{\rr}{\ensuremath{\mathbb{R}}}
\newcommand{\zz}{\ensuremath{\mathbb{Z}}}
\newcommand{\qq}{\ensuremath{\mathbb{Q}}}
\newcommand{\nn}{\ensuremath{\mathbb{N}}}
\newcommand{\ff}{\ensuremath{\mathbb{F}}}
\newcommand{\cc}{\ensuremath{\mathbb{C}}}
\newcommand{\ee}{\ensuremath{\mathbb{E}}}
\renewcommand\O{\ensuremath{\emptyset}}
\newcommand{\norm}[1]{{\left\lVert{#1}\right\rVert}}
\newcommand{\ve}[1]{{\mathbf{#1}}}
\newcommand\allbold[1]{{\boldmath\textbf{#1}}}
\DeclareMathOperator{\lcm}{lcm}
\DeclareMathOperator{\im}{im}
\DeclareMathOperator{\coim}{coim}
\DeclareMathOperator{\dom}{dom}
\DeclareMathOperator{\tr}{tr}
\DeclareMathOperator{\rank}{rank}
\DeclareMathOperator*{\var}{Var}
\DeclareMathOperator*{\ev}{E}
\DeclareMathOperator{\sinc}{sinc}
\DeclareMathOperator{\dg}{deg}
\DeclareMathOperator{\aff}{aff}
\DeclareMathOperator{\conv}{conv}
\DeclareMathOperator{\epi}{epi}
\DeclareMathOperator{\inte}{int}
\DeclareMathOperator{\ri}{ri}
\DeclareMathOperator*{\argmin}{argmin}
\DeclareMathOperator*{\argmax}{argmax}
\DeclareMathOperator{\graph}{graph}
\DeclareMathOperator{\sgn}{sgn}
\DeclareMathOperator*{\Rep}{Rep}
\DeclareMathOperator{\Proj}{Proj}
\DeclareMathOperator{\prox}{prox}
\DeclareMathOperator{\mat}{mat}
\let\vec\relax
\DeclareMathOperator{\vec}{vec}
\let\Re\relax
\DeclareMathOperator{\Re}{Re}
\let\Im\relax
\DeclareMathOperator{\Im}{Im}
% Put x \to \infty below \lim
\let\svlim\lim\def\lim{\svlim\limits}

%wide hat
\usepackage{scalerel,stackengine}
\stackMath
\newcommand*\wh[1]{%
\savestack{\tmpbox}{\stretchto{%
  \scaleto{%
    \scalerel*[\widthof{\ensuremath{#1}}]{\kern-.6pt\bigwedge\kern-.6pt}%
    {\rule[-\textheight/2]{1ex}{\textheight}}%WIDTH-LIMITED BIG WEDGE
  }{\textheight}% 
}{0.5ex}}%
\stackon[1pt]{#1}{\tmpbox}%
}
\parskip 1ex

%Make implies and impliedby shorter
\let\implies\Rightarrow
\let\impliedby\Leftarrow
\let\iff\Leftrightarrow
\let\epsilon\varepsilon

% Add \contra symbol to denote contradiction
\usepackage{stmaryrd} % for \lightning
\newcommand\contra{\scalebox{1.5}{$\lightning$}}

% \let\phi\varphi

% Command for short corrections
% Usage: 1+1=\correct{3}{2}

\definecolor{correct}{HTML}{009900}
\newcommand\correct[2]{\ensuremath{\:}{\color{red}{#1}}\ensuremath{\to }{\color{correct}{#2}}\ensuremath{\:}}
\newcommand\green[1]{{\color{correct}{#1}}}

% horizontal rule
\newcommand\hr{
    \noindent\rule[0.5ex]{\linewidth}{0.5pt}
}

% hide parts
\newcommand\hide[1]{}

% si unitx
\usepackage{siunitx}
\sisetup{locale = FR}

%allows pmatrix to stretch
\makeatletter
\renewcommand*\env@matrix[1][\arraystretch]{%
  \edef\arraystretch{#1}%
  \hskip -\arraycolsep
  \let\@ifnextchar\new@ifnextchar
  \array{*\c@MaxMatrixCols c}}
\makeatother

\renewcommand{\arraystretch}{0.8}

% Environments
\makeatother
% For box around Definition, Theorem, \ldots
%%fakesection Theorems
\usepackage{thmtools}
\usepackage[framemethod=TikZ]{mdframed}

\theoremstyle{definition}
\mdfdefinestyle{mdbluebox}{%
	roundcorner = 10pt,
	linewidth=1pt,
	skipabove=12pt,
	innerbottommargin=9pt,
	skipbelow=2pt,
	nobreak=true,
	linecolor=blue,
	backgroundcolor=TealBlue!5,
}
\declaretheoremstyle[
	headfont=\sffamily\bfseries\color{MidnightBlue},
	mdframed={style=mdbluebox},
	headpunct={\\[3pt]},
	postheadspace={0pt}
]{thmbluebox}

\mdfdefinestyle{mdredbox}{%
	linewidth=0.5pt,
	skipabove=12pt,
	frametitleaboveskip=5pt,
	frametitlebelowskip=0pt,
	skipbelow=2pt,
	frametitlefont=\bfseries,
	innertopmargin=4pt,
	innerbottommargin=8pt,
	nobreak=false,
	linecolor=RawSienna,
	backgroundcolor=Salmon!5,
}
\declaretheoremstyle[
	headfont=\bfseries\color{RawSienna},
	mdframed={style=mdredbox},
	headpunct={\\[3pt]},
	postheadspace={0pt},
]{thmredbox}

\declaretheorem[%
style=thmbluebox,name=Theorem,numberwithin=section]{thm}
\declaretheorem[style=thmbluebox,name=Lemma,sibling=thm]{lem}
\declaretheorem[style=thmbluebox,name=Proposition,sibling=thm]{prop}
\declaretheorem[style=thmbluebox,name=Corollary,sibling=thm]{coro}
\declaretheorem[style=thmredbox,name=Example,sibling=thm]{eg}

\mdfdefinestyle{mdgreenbox}{%
	roundcorner = 10pt,
	linewidth=1pt,
	skipabove=12pt,
	innerbottommargin=9pt,
	skipbelow=2pt,
	nobreak=true,
	linecolor=ForestGreen,
	backgroundcolor=ForestGreen!5,
}

\declaretheoremstyle[
	headfont=\bfseries\sffamily\color{ForestGreen!70!black},
	bodyfont=\normalfont,
	spaceabove=2pt,
	spacebelow=1pt,
	mdframed={style=mdgreenbox},
	headpunct={ --- },
]{thmgreenbox}

\declaretheorem[style=thmgreenbox,name=Definition,sibling=thm]{defn}

\mdfdefinestyle{mdgreenboxsq}{%
	linewidth=1pt,
	skipabove=12pt,
	innerbottommargin=9pt,
	skipbelow=2pt,
	nobreak=true,
	linecolor=ForestGreen,
	backgroundcolor=ForestGreen!5,
}
\declaretheoremstyle[
	headfont=\bfseries\sffamily\color{ForestGreen!70!black},
	bodyfont=\normalfont,
	spaceabove=2pt,
	spacebelow=1pt,
	mdframed={style=mdgreenboxsq},
	headpunct={},
]{thmgreenboxsq}
\declaretheoremstyle[
	headfont=\bfseries\sffamily\color{ForestGreen!70!black},
	bodyfont=\normalfont,
	spaceabove=2pt,
	spacebelow=1pt,
	mdframed={style=mdgreenboxsq},
	headpunct={},
]{thmgreenboxsq*}

\mdfdefinestyle{mdblackbox}{%
	skipabove=8pt,
	linewidth=3pt,
	rightline=false,
	leftline=true,
	topline=false,
	bottomline=false,
	linecolor=black,
	backgroundcolor=RedViolet!5!gray!5,
}
\declaretheoremstyle[
	headfont=\bfseries,
	bodyfont=\normalfont\small,
	spaceabove=0pt,
	spacebelow=0pt,
	mdframed={style=mdblackbox}
]{thmblackbox}

\theoremstyle{plain}
\declaretheorem[name=Question,sibling=thm,style=thmblackbox]{ques}
\declaretheorem[name=Remark,sibling=thm,style=thmgreenboxsq]{remark}
\declaretheorem[name=Remark,sibling=thm,style=thmgreenboxsq*]{remark*}

\theoremstyle{definition}
\newtheorem{claim}[thm]{Claim}
\theoremstyle{remark}
\newtheorem*{case}{Case}
\newtheorem*{notation}{Notation}
\newtheorem*{note}{Note}
\newtheorem*{motivation}{Motivation}
\newtheorem*{intuition}{Intuition}

% Make section starts with 1 for report type
%\renewcommand\thesection{\arabic{section}}

% End example and intermezzo environments with a small diamond (just like proof
% environments end with a small square)
\usepackage{etoolbox}
\AtEndEnvironment{vb}{\null\hfill$\diamond$}%
\AtEndEnvironment{intermezzo}{\null\hfill$\diamond$}%
% \AtEndEnvironment{opmerking}{\null\hfill$\diamond$}%

% Fix some spacing
% http://tex.stackexchange.com/questions/22119/how-can-i-change-the-spacing-before-theorems-with-amsthm
\makeatletter
\def\thm@space@setup{%
  \thm@preskip=\parskip \thm@postskip=0pt
}

% Fix some stuff
% %http://tex.stackexchange.com/questions/76273/multiple-pdfs-with-page-group-included-in-a-single-page-warning
\pdfsuppresswarningpagegroup=1

\renewcommand{\baselinestretch}{1.5}
\RequirePackage{hyperref}[6.83]
\hypersetup{
  colorlinks=false,
  frenchlinks=false,
  pdfborder={0 0 0},
  naturalnames=false,
  hypertexnames=false,
  breaklinks
}
\urlstyle{same}

\usepackage{graphics}
\usepackage{epstopdf}

%%
%% Add support for color in order to color the hyperlinks.
%% 
\hypersetup{
  colorlinks = true,
  allcolors = siaminlinkcolor,
  urlcolor = siamexlinkcolor,
}
%%fakesection Links
\hypersetup{
    colorlinks,
    linkcolor={red!50!black},
    citecolor={green!50!black},
    urlcolor={blue!80!black}
}
%customization of cleveref
\RequirePackage[capitalize,nameinlink]{cleveref}[0.19]

% Per SIAM Style Manual, "section" should be lowercase
\crefname{section}{section}{sections}
\crefname{subsection}{subsection}{subsections}
\Crefname{section}{Section}{Sections}
\Crefname{subsection}{Subsection}{Subsections}

% Per SIAM Style Manual, "Figure" should be spelled out in references
\Crefname{figure}{Figure}{Figures}

% Per SIAM Style Manual, don't say equation in front on an equation.
\crefformat{equation}{\textup{#2(#1)#3}}
\crefrangeformat{equation}{\textup{#3(#1)#4--#5(#2)#6}}
\crefmultiformat{equation}{\textup{#2(#1)#3}}{ and \textup{#2(#1)#3}}
{, \textup{#2(#1)#3}}{, and \textup{#2(#1)#3}}
\crefrangemultiformat{equation}{\textup{#3(#1)#4--#5(#2)#6}}%
{ and \textup{#3(#1)#4--#5(#2)#6}}{, \textup{#3(#1)#4--#5(#2)#6}}{, and \textup{#3(#1)#4--#5(#2)#6}}

% But spell it out at the beginning of a sentence.
\Crefformat{equation}{#2Equation~\textup{(#1)}#3}
\Crefrangeformat{equation}{Equations~\textup{#3(#1)#4--#5(#2)#6}}
\Crefmultiformat{equation}{Equations~\textup{#2(#1)#3}}{ and \textup{#2(#1)#3}}
{, \textup{#2(#1)#3}}{, and \textup{#2(#1)#3}}
\Crefrangemultiformat{equation}{Equations~\textup{#3(#1)#4--#5(#2)#6}}%
{ and \textup{#3(#1)#4--#5(#2)#6}}{, \textup{#3(#1)#4--#5(#2)#6}}{, and \textup{#3(#1)#4--#5(#2)#6}}

% Make number non-italic in any environment.
\crefdefaultlabelformat{#2\textup{#1}#3}

% My name
\author{Jaden Wang}



\begin{document}
\subsection{Existence and uniqueness of minimizers}
\subsubsection{Existence}
\begin{intuition}
	We want to generalize EVT to functions in $ \Gamma_0(\rr^{n})$.
\end{intuition}
\begin{defn}[coercive]
	$ f: \rr^{n} \to [-\infty,\infty]$ is \allbold{coercive} if
	\[
		\lim_{ \norm{ x}  \to \infty} f(x) = \infty
	.\] 
\end{defn}
\begin{note}
In other words, it grows!
\end{note}
\begin{prop}
	$ f$ is coercive iff all sub-level sets  $ \{x: f(x) \leq \alpha\} $ are bounded.
\end{prop}
\begin{intuition}
	There is no "horizontal asymptote".
\end{intuition}
\begin{prop}
	If $ f \in \Gamma_0 (\rr^{n})$, $ f$ is coercive iff there exists  $ \alpha$ s.t. $ \{x:f(x) \leq \alpha\} $ is non-empty and bounded.
\end{prop}

\begin{remark}
Convex + coercive need not be coercive.
\end{remark}

\begin{thm}[existence]
	Let $ C$ be closed, convex, $ f \in \Gamma_0(\rr^{n}), C \cap \dom(f) \neq \O$, then $ \min_{x \in C} f(x) $ exists if either
	\begin{enumerate}[label=\arabic*)]
		\item $ f$ is coercive or
		\item $ C$ is bounded.
	\end{enumerate}
\end{thm}

\subsubsection{Uniqueness}
~\begin{thm}[Uniqueness]
	Let $ C$ be convex,  $ f$ is proper and convex, $ C \cap \dom(f) \neq \O$, then there is at most one minimizer of $ \min_{x \in C} f(x)$ if either
	\begin{enumerate}[label=\arabic*)]
		\item $ f$ is strictly convex or
		\item  $ C \cap  \argmin f = \O$ and $ C$ is strictly convex.
	\end{enumerate}
\end{thm}
\begin{intuition}
	If the unconstrained minimizer is not in the constrained set, then $ C$ and not $ f$ dictates uniqueness.
\end{intuition}

\begin{defn}[strictly convex set]
A set $ C$ is  \allbold{strictly convex} if there are no line segments on its boundary. That is,
\[
	\frac{x+y}{ 2} \in \inte (C) \ \forall \ x,y \in C, x\neq y
.\] 
\end{defn}
\begin{remark}
	In summary: let $ f \in \Gamma_0(\rr^{n})$. If $ f$ is strictly convex  $ \implies$ at most 1 minimizer. If $ f$ is coercive  $ \implies$ at least 1 minimizer.
\end{remark}
\begin{coro}
If $ f$ is strongly convex, then there exists a unique minimizer.
\end{coro}
\begin{proof}
Strongly convex $ \implies $ strictly convex. Also notice that if we choose $ x=0$, then
 \[
	 f(y)\geq f(x) + \langle \nabla f(x),y-x \rangle + \frac{\mu}{ 2} \norm{ y-x}^2 
\]
will be "coerced" to go to infinity as $ \norm{ y} \to \infty $.
\end{proof}

\subsection{Proximity operator}
~\begin{defn}[orthogonal projection]
The \allbold{orthogonal projection} of a point $ y$ onto a set  $ C$ is
 \[
	 \Proj_C(y) = \argmin_{x \in C} \norm{ x-y}_2 = \argmin_{x \in C} \frac{1}{2} \norm{ x-y}_2^2  
.\]
\end{defn}
\begin{note}
Applying a monotone transformation to the objective doesn't change the location of minimizer, hence we can use norm squared.
\end{note}
\begin{remark}
The solution to the projection exists and is unique if $ C$ is a Chebyshev set  \emph{i.e.} closed and convex.
\end{remark}
\begin{defn}[proximity operator]
	The \allbold{proximity operator} of a function $ f \in \Gamma_0 (\rr^{n})$ is
	\[
		\prox_f (y) = \argmin_x \left(\frac{1}{2} \norm{ x-y}_2^2 + f(x) \right)
	.\] 
\end{defn}
\begin{ques}
Is this minimizer unique? Does it even exist?
\end{ques}

The solution is unique because $ f$ is convex and the norm squared is strongly convex in $ x$, making the sum strongly convex which guarantees a unique minimizer.

\begin{eg}
Let $ f = I_C$ be an indicator function, then
 \[
	 \prox_f(y) = \Proj_C(y)
.\] 
\end{eg}
\begin{remark}
We often insert a scaling constant for convenience. This constant turns out to be the step size in an iterative algorithm:
\[
	\prox_{tf}(y) = \argmin_x \frac{1}{2}\norm{ x-y}_2 ^2 + t f(x)
.\] 
\end{remark}
\begin{eg}
	$ f(x) = \frac{1}{2} \norm{ x}_2^2 $. Then
	\begin{align*}
		\prox_f(y) &= \argmin_x \ \frac{1}{2} \norm{ x-y}_2 ^2 + \frac{t}{2} \norm{ x}_2 ^2\\
		\text{ Fermat and differentiable}\implies 0 &= x^* -y + tx^*  \\
		x^*  &= (1+t)^{-1} y
	\end{align*}
\end{eg}
\begin{eg}[taking gradient]
	$ f(x) = \frac{1}{2} \norm{ Ax-b}^2 $, what is $ \nabla f$?

	Think of it as $ f = g \circ h$, where  $ g(x) = \frac{1}{2} \norm{ x}^2 $. 
\end{eg}

\begin{eg}
	$ f(x) = \norm{ x}_1 $. So
	\begin{align*}
		\prox_{tf} (y) &= \argmin_x \frac{1}{2}\norm{ x-y}_2 ^2 + t\norm{ x}_1 \\
			       &= \argmin_x \frac{1}{2} \sum_{ i= 1}^{ n} (x_i-y_i)^2 + t \sum_{ i= 1}^{ n} |x_i|
	\end{align*}
	Notice each $ i$ is separable, so WLOG we just need to find the optimal $ x_i$, dropping the $ i$ in notation:
	 \begin{align*}
		 \argmin_x \ \frac{1}{2}(x-y)^2 + t|x| 
	\end{align*}
	By Fermat's rule,
	\begin{align*}
		0 &\in \partial \left(\frac{1}{2}(x-y)^2 + t|x|\right)  \\
		0 &\in x-y + t \partial (|x|) \qquad \qquad    \text{ via CQ such as full domain} \\
		0 &\in x-y + \begin{cases}
			t & x > 0\\
			[-t,t] & x = 0\\
			-t & x<0
		\end{cases} 
	\end{align*}
	\begin{case}[1]
		If $ x>0$, we have
		 \begin{align*}
			0 = x-y + t \implies x = y-t
		\end{align*}
		This is only valid if $ y-t > 0 \implies y > t$.
	\end{case}
	\begin{case}[2]
	If $ x<0$, we have  $ 0 = x-y-t \implies x = y+ t \implies y < -t$.
	\end{case}
	\begin{case}[3]
		If $ x = 0$, we have  $ 0 \in -y + [-t,t] \implies y \in [-t,t]$. 
	\end{case}
	This perfectly covers all cases.
	\begin{align*}
		\prox_{t|x|} (y) &=
		\begin{cases}
			y-t & y> t\\
			0 & y \in [-t,t]\\
			y+t & y<-t
		\end{cases}\\
				 &= \sgn(y) \cdot \lfloor |y|-t \rfloor_{+} \qquad  \text{ where } \lfloor \alpha \rfloor_+ = \max (\alpha,0) 
	\end{align*}
	~\begin{figure}[H]
		\centering
		\includegraphics[width=0.7\textwidth]{./figures/soft_thresholding.png}
	\end{figure}
	The last expression is called \allbold{soft thresholding} and is more computationally efficient. Therefore, $ \prox_{t \norm{ \cdot }_1 }$ is just componentwise soft-thresholding.
\end{eg}

\subsubsection{Rules}
If $ \prox_f,\prox_g$ are known, there is not a general formula for  $ \prox_{f+g}$. Even $ \prox_{f\circ L}$ isn't easy.

\subsection{Moreau envelope}
See supplemental lecture notes in "Proximity Operator".
\end{document}
